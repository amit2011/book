% ##################################################################################################################
\chapter{Sioux Falls}
\label{ch:siouxfalls}
\hfill \textbf{Author:} Artem Chakirov

\editdone{This text has undergone the professional edit. Please no grammatical changes anymore! They are most-probably wrong.}

% ##################################################################################################################
The Sioux Falls scenario provided a convenient test-case, combining fully dynamic demand fitted with realistic socio-economic and demographic attributes with a small-scale road network including an integrated public transportation system. Based on the Sioux Falls road network commonly used for tests and demonstration purposes in transportation literature \citep[][]{BarGera_TNTP_Webpage_2013}, it allowed quick and convenient experiments on new policy or software implementations with \gls{matsim} on a heterogeneous agent population, with a high degree of spatial resolution, but without significant computational requirements. However, it is important to stress that, despite the use of real world data for the generation of the enriched Sioux Falls scenario, it did not aim to replicate the real City of Sioux Falls in South Dakota, US and remains a fictitious test case. Detailed report on scenario generation and its characteristics is provided by \citet[][]{ChakirovFourie_TechRep_FCL_2014} and can also be found at \url{www.matsim.org/scenario/sioux-falls}. 

% ====================================================================================================
\section{Demand}
A realistic, socio-economically and demographically diverse demand population---with  heterogeneous use preferences---was crucial for unlocking the full potential of an agent-based simulation like \gls{matsim}. However, generation of a disaggregated demand description on individual and household levels close to reality was challenging; not only for trip origins and destinations, but also with respect to travel pattern relation and socio-demographic travelers' characteristics.

To address this challenge for the Sioux Falls scenario, and represent the household structure, demographic profile and income distribution as realistically as possible, a synthetic household population, using the \citet[][]{BarGeraEtAl_TRB_2009} entropy optimization technique, was generated. It matched the aggregate distribution of demographic attributes (age, sex and household income) recorded during the 2010 US Census. It contained census tracts inside, and adjoining, the city center of Sioux Falls and was composed of household and person records taken from the (anonymous) 5-year American Community Survey sample (2007-2011), covering 5\,\% of all households.

To keep the scenario accessible, as well as facilitating interpretation and understanding of possible effects on policies studied, only two simple activity chains were initially included: ``home – work – home'' and ``home – other – home''. Activity locations were identified using building stock data set provided by the City of Sioux Falls \gls{gis} division. Each household's home location was assigned randomly to a residential unit within the household's tract. Because no information on the real number and distribution of work places within the relevant area was easily accessible, the static \gls{od} matrix from \citet[][]{LeBlancEtAl_TransRes_1975} was taken as a workplace attraction indicator for each zone. Then, assignment of work places to individual workers, as well as locations of secondary (other) activities, was performed using a parameter-free radiation model presented by \citet[][]{SiminiEtAl_NAT_2012}.

To exploit the full potential of disaggregated demand and add another degree of realism to the scenario, car ownership on the household level was modeled using an ordered probit model, presented by \citet[][]{GiulianoDargay_TransResA_2006} and based on the \gls{npts}~1995. In addition to socio-demographic household characteristics (number of adults, children, pensioners, household income), the model used residential location attributes (population density, public transport access and dwelling type), which better described specific Sioux Falls scenario characteristics, as well as its area-wide bus network. 

% ====================================================================================================
\section{Supply} 
A realistic transportation test network should ensure sufficient complexity of travelers’ choice dimensions while limiting  computational effort. To this end, the Sioux Falls test network was introduced by \citet[][]{MorlokEtAl_ResRep_org-fhwa_1973} and later adapted as a benchmark and test scenario in many publications (see \citet[][]{ChakirovFourie_TechRep_FCL_2014} for overview). The network structure captured the major arterial roads of Sioux Falls, South Dakota, but was never intended to replicate the real city, or all characteristics of its transportation system, such as travel times or modal split. The original network was comprised of 76\,arcs, 24\,nodes and 552\,\gls{od} pairs. For this scenario, road capacities were adjusted according to values provided by the Highway Capacity Manual \citet[][]{HCM_2010} and other related research publications \citep[e.g.,][]{NgCFSmall_Transportation_2012}. The public transportation network added to the scenario included five bus lines, as initially proposed by \citet[][]{AbdulaalLeBlanc_TransScience_1979}, with bus stops placed at regular intervals of 600\,meters. 

Due to the design of \gls{matsim}’s queue simulation, agents were handled only at the beginning and end of each network link and could not enter or leave a link along its length. Therefore, origins and destinations located along very long links led to spatial detail loss, as all origins and destinations along the length of the link were effectively assigned the same coordinates. Consequently, to improve spatial detail level, all links of the Sioux Falls network were evenly split into smaller links, with maximum length of 500\,meters each. Following this operation, number of nodes was increased to 282 and number of links to 334, without changing effective network topology.

In addition to car and bus modes, walking as ``\gls{teleported}'' mode, with constant travel speeds, and with no interaction with other users, is used as the non-motorized transportation mode. 

% ====================================================================================================
\section{Behavioral Parameters}
Behavioral parameters used in utility functions were based on estimated demand model for Sydney by \citet[][]{TirachiniHensherRose_TransResB_2014}. Before applying parameters in an activity-based context, time-related parameters had to be adjusted to account for utility gained from activity performance. Thus, to provide a value for marginal utility of performing an activity, the travel mode with smallest the disutility was set as a baseline, under the assumption that traveling with this mode was equivalent to idling/doing nothing. Corresponding parameters were split into opportunity costs of time and a mode-specific disutility of traveling, as has in previous \gls{matsim}-related publications as \citet[e.g,][]{KickhoeferEtAl2011PolicyEvaluationIncome}. 

% ====================================================================================================
\section{Results, Drawbacks and Outlook}
Sioux Falls scenario stability and performance was tested using two sets of activity timing constraints, as well as five different random seeds, which all delivered stable and realistic results. \citet[][]{ChakirovFourie_TechRep_FCL_2014} also investigated \gls{mfd} existence and hysteresis characteristics, as discussed in \citet[][]{GeroliminisDaganzo_TRB_2007, GeroliminisDaganzo_TransResB_2008, GeroliminisSun_TransResA_2011}. 

However, recent experience has shown certain coarse network drawbacks; it represented only major arterial roads and neglected minor neighborhood and collector road links. With an elaborate synthetic population and high rush hour demand peaks, the network seemed to be sensitive to network breakdowns under high loading conditions. 

Along with the coarse road network, the coarse public transport network level and the resulting  low level of accessibility (for parts of the population) represented another drawback, particularly relevant to simulation and evaluation of policies sensitive to, or requiring, a certain share of public transport users. 

Replacing the original Sioux Falls network with a finer network obtained from the crowd-sourced \gls{osm} and adding additional public transport lines would address the above-mentioned scenario weakness. However, this introduces a different set of drawbacks and would require further attention. First, the significantly larger number of network links and nodes increases time and resources for routing and dynamic queue simulation and could erase the advantages of a small-scale network. Extended simulation times can be tackled with the new pseudo-simulation methodology, currently developed by \citet[][]{FourieEtAl_TRR_2013}.
Second, total network capacity increase leads to reduction or even disappearance of congestion during peak hours, although including freight and through traffic in the scenario can make it more realistic and address congested conditions during peak-hours. 

% ##################################################################################################################
