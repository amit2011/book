% ##################################################################################################################
\chapter{Barcelona}
\label{ch:barcelona}
\hfill \textbf{Author:} Miguel Picornell, Maxime Lenormand

\editdone{This text has undergone the professional edit. Please no grammatical changes anymore! They are most-probably wrong.}

% ##################################################################################################################
The Barcelona scenario is one of the three case studies (together with London and Zürich) carried out under the framework of the \gls{eunoia} project. The main goal of the Barcelona case study was to evaluate the impact of different public bike-sharing schemes in the city. The study area covers the metropolitan Barcelona area , with special focus on the city center, where public bike-sharing stations are located. For this stud a novel bike-sharing module was developed by \gls{eth} Zürich.

% ===================================================================================
\section{Transport Supply: Network and Public Transport}
The road network was extracted from the \gls{transcad} model used by the city of Barcelona. Public transport supply was also considered, comprising: bus, underground, tram, train and bike-sharing. Information about stops and schedules was obtained from the public information available at the Barcelona Open Data platform, as well as from the Barcelona transport authority website. 

% ===================================================================================
\section{Transport Demand: Population} 
Agent plans were defined using anonymised mobile phone registers \glspl{cdr}. From mobile phone data, it is possible to identify places where agents perform activities and corresponding trips. Activities have been classified as ``home'', ``work'' and ``other'' (including as ``other'', ``leisure'', ``shopping'', etc.). A sample of around 15\,\% of the population was used in the simulation. Modes used  in the simulation model include: walking, cycling, public transport and car.

% ===================================================================================
\section{Calibration and Validation}
Different data sources were used to calibrate and validate the model. First, to validate agent plans obtained from mobile phone data, results were compared to \gls{emef}, indicating that mobile phone data provides information similar to traditional surveys. Additionally, agents' utility function was calibrated using the modal split from \gls{emef} and road counts.

% ===================================================================================
\section{Results and More Information}
At the time this summary was written, the calibration process was still ongoing. More detailed information about the scenario and main results can be found at:
\url{www.eunoia-project.eu/publications/} (project deliverables/Report on Case Study 3: Barcelona).

% ##################################################################################################################
