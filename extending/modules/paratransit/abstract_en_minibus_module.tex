% !TeX root = ../../../main.tex
Paratransit is an informal, market-oriented, and self-organizing public transport system. Despite the great importance of this transport mode it is mainly unsubsidized and only relies on the collected fares. 
Paratransit systems can be categorized by route pattern and function, by
organization of drivers, kind of stops, and fare type. Most case studies covered by the thesis of \citet[][]{Neumann_PhDThesis_2014} indicate that
paratransit services are mainly organized as route associations operating 8-15
seater vans on fixed routes. Most of the services run in direct competition to a
public transport system of a public transit authority. Such a service---minibuses with fixed routes but without fixed schedule---is often called a jitney service.
The minibus module of MATSim is based on those most common characteristics with the understanding that the jitney/minibus
service is one out of many possible paratransit services.

The minibus model is integrated in the multi-modal multi-agent simulation of MATSim. In the model, competing minibus operators start exploring the public transport market offering their services. With more successful operators expanding and less successful operators going bankrupt, a sustainable network of minibus services evolves. In \citet[][]{Neumann_PhDThesis_2014}, the model is verified through multiple illustrative scenarios that analyze the model's sensitivity towards different demand patterns, transfers, and the interactions of minibuses and a formal operator's fixed train~line.

The minibus model can be applied to two different fields of transport planning. First, there is the simulation of real paratransit that aims to help understand the implications that lie within the relationship of the different paratransit stakeholders. The model is able to create ``close-to-reality'' minibus networks in a South African context. \citet[][]{NeumannEtAl2014MinibusRSA} give an in-depth presentation of the application of the module and paratransit in South Africa in general. Given the informal and emergent nature of minibus paratransit in developing countries, routes, schedules and fares are usually not published and is only captured in the tacit knowledge of operators and frequent users. Applying the minibus model has proven valuable in getting a better understanding of how routes evolve. Instead of imposing routes and schedules \emph{on} the MATSim model, as is appropriately the case for formal transit, the modeler can observe and get the paratransit routes as an output \emph{from} the model. As each operator aims to maximize their profit, the resulting network often favors the business objectives of the operators, and not necessarily the connectivity and mobility of the mode's users. This feature of the model accurately captures route forming behavior in the South African case where commuters are often required to take multiple, longer trips instead of direct trips.

Second, the same model provides a demand-driven approach to solve the network design problem of a formal transit authority. Thus, it can be used as a planning tool for the optimization of single transit lines or networks. For more details on the second form of application see Section~\ref{sec:paratransit_application}.

For further reading: \citet[][]{Neumann_PhDThesis_2014} provides an understanding of the underlying principles of paratransit services, namely minibus services, its stakeholders, fares, route functions, and patterns. Furthermore, it contains an in-depth description of the minibus model, its theoretical background, and its application to illustrative scenarios as well as real world examples. The website of MATSim also hosts the documentation of the latest implementation at \url{http://matsim.org/doxygen}.



