% !TeX root = ../../../main.tex
\section{Network Planning or solving the transit network design problem with MATSim}
\label{sec:paratransit_application}

\ah{Hilft da wagonSim oder dass was die IVT.Weidmann-Gruppe gerade macht?}\an{Braucht's eigentlich nicht aber man kann die beiden Koppeln und entsprechend auch Gueterfluesse optimieren.}

The success of a public transport system highly depends on its network design. When transport companies try to optimize a line with respect to running costs there is also the demand to be taken into consideration. The best cost structure will not be sustainable if potential customers leave the system and opt for alternatives like private cars. The basic problem to solve is to find sustainable transit lines which offer the best service possible for the~customer.

More specifically
\begin{itemize}
\item The demand side of the customers asks for direct hassle-free connections.
\item The supply side of the operators asks for profitable lines to operate.
\end{itemize}
Examples of a market-oriented and moreover self-organizing public transport system are informal public transit systems around the world. These services are often referred to as paratransit. For an in-depth coverage of paratransit the reader is referred to Section~\ref{sec:paratransit} and the references therein. Despite the great importance of this transport mode it is mainly unsubsidized and only relies on the collected fares. Thus, the knowledge on paratransit and its ability to identify and fill market niches with self-supporting transit services provides an interesting approach to solve the network design problem of a formal public transit company.

\noindent
Accordingly, the minibus module of MATSim (Section~\ref{sec:paratransit}) provides a demand-driven approach to solve the network design problem of a formal transit authority. Thus, it can be used as a planning tool for the optimization of single transit lines or networks. In the thesis of \citet[][]{Neumann_PhDThesis_2014}, the model is applied to two different planning problems of the public transit authority of Berlin, BVG. In the first scenario, the model constructs a transit system from scratch for the district of Steglitz-Zehlendorf. The second scenario analyzes the impact of the closure of Tegel airport on BVG's bus network. Apart from Tegel itself, the rest of the bus network is found to be unaffected by the closure of the airport. The resulting transit system of the minibus model resembles the changes BVG had scheduled for the closure of Tegel.

In conclusion, the minibus model developed in the thesis automatically adapts the supply to the demand. The model does not only grow networks from scratch but can also test for an existing transit line's sustainability and can further optimize the line regarding its frequency, its time of operation, its length, and its route. Again, the optimization process is fully integrated into the behavior-rich environment of the multi-agent simulation of MATSim reflecting the reactions from the passengers as well as from competing transit services and other road users. Thus, the minibus model can be used along with more complex scenarios like city-wide tolls or pollution analyses.




