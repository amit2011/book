\chapter{Miscellaneous Contribs in Short}
\label{ch:misccontribs}
% ##################################################################################################################

\hfill \textbf{Author:} Kai Nagel

% ##################################################################################################################
\section{Analysis}
\label{sec:contrib-analysis}

\createStandardInformationBasic{{\url{http://matsim.org/javadoc} $\to$ analysis}}{%
%
No standard invocation. See the \lstinline|RunKNEventsAnalyzer| class for intuition.
%
}{%
%
Analysis tools without standard configuration.
%
}{%
%
none
%
}

%% \lstinline{org.matsim.contribs.analysis.*}

This \gls{contribution} collects various analysis tools for \gls{matsim} output.  

One important reason for having this in a \gls{contribution} rather than in a playground is the \gls{maven} layout of the repository: \Glspl{contribution} can use material from other \glspl{contribution}, but not from the playgrounds. 
In consequence, analysis tools that are needed in a \gls{contribution} needs to be in a \gls{contribution} themselves. 
The analysis \gls{contribution} is a possible place where to put them.

% ##################################################################################################################
\section{freightChainsFromTravelDiaries}
\label{sec:freightChainsFromTravelDiaries}
\createStandardInformationBasic{{\url{http://matsim.org/javadoc} $\to$ freightChainsFromTravelDiaries}}{Currently not possible.}{No standard configuration.}{
%
\cite{Schneider2011PhD}
%
}
Sebastian Schneider has done a Ph.D.\ dissertation about generating freight vehicle chains by essentially re-sampling the information contained in the German survey \gls{kid} \citep{SteinmeyerWagner2005KiD}.  
Since the \gls{kid} is essentially an activity-based travel diary, the method should also be applicable to other situations.
Since Sebastian has left science for the time being, he allowed us to take his code and integrate it into the repository, under the \gls{gpl}. For the time being, it will just ``sit'' here until someone attempts to make it work.

%##################################################################################################################
%\section{gtfs2matsimtransitschedule}
%\label{sec:gtfs2matsimtransitschedule}
%
%\kai{todo not kai}

%\ah{wird in in Section~\ref{sec:SemiTool} abgehandelt}

% ##################################################################################################################
\section{matrix based pt router}
\label{sec:matrix-based-pt-router}
\ah{why not createStandardInfo?}
\createStandardInformationBasic{%
%
\url{http://matsim.org/javadoc} $\to$ matrixbasedptrouter
%
}{%
%
\url{http://matsim.org/javadoc} $\to$ matrixbasedptrouter $\to$ \lstinline|RunMatrixBasedPTRouterExample| class
%
}{%
%
\lstinline{matrixBasedPtRouter} section of the \gls{configfile} (only available when the matrixbasedptrouter \gls{contribution} is correctly loaded)
%
}{%
%
Section~3.1 of \citet{NicolaiNagel2013HandbookIntegrationOfTransportLanduse}; \citet{RoederNagel2013SketchPlanningBrussels}
%
}

The matrix based \gls{pt} router reads a list of \gls{pt} stops, and constructs ``teleported'' \gls{pt} routes using the stops nearest to origin and destination.  That is, each resulting trip will approximately look as follows:
\begin{xml}
<act type="previous" ... />
<!-- begin trip -->
<leg mode="walk" ... />
<act type="ptInteraction" ... />
<leg mode="pt" ... />
<act type="ptInteraction" ... />
<leg mode="walk" ... />
<!-- end trip -->
<act type="next" ... />  
\end{xml}

The attributes of the walk and the \gls{pt} legs will be computed from the coordinates of the locations in the same way as teleportation routing (see Section~\ref{sec:teleportation-routing}), and then taken at face value in the \gls{mobsim} (see Section~\ref{sec:teleportation-qsim}).

Travel times and travel distances between \gls{pt} stops can alternatively be given by corresponding matrices.  This is particularly useful if a \gls{pt} assignment exists and such information can be extracted from that.  This was used by \citet{RoederNagel2013SketchPlanningBrussels} and by \citet{ZoelligRenner_PhDThesis_2014}.

% ##################################################################################################################
\section{MATSim4UrbanSim}
\label{sec:matsim4urbansim}

\editdone{This text has undergone the professional edit. Please no grammatical changes anymore! They are most-probably wrong.}

%% \subsection{Basic Information}
%% \label{sec:matsim4urbansim-stdInfo}

\createStandardInformationBasic{%
\url{http://matsim.org/javadoc} $\to$ matsim4urbansim
%
}{%
%
The module is invoked from a live UrbanSim implementation.
%
}{%
%
In the UrbanSim config file.
%
}{%
\citet[][]{Nicolai2011Couplinganurbansimulationmodelwithatravelmodel}; \citet{NicolaiNagelHiResAccessibilityMethod}; \citet{NicolaiNagel2013HandbookIntegrationOfTransportLanduse}
%
}

% ##################################################################################################################
%% \subsection{Summary}
``MATSim4UrbanSim'' is an adapter package for using \gls{matsim} as a travel model plug-in to \gls{urbansim}, a well-known land use simulation \citep[e.g.][see \url{http://www.urbansim.org}]{WaddellEtc2003UrbanSim}.
\gls{urbansim} has, for example, submodels for residential location choice, commercial location choice, or development and building construction, thus creating synthetic  potential urban or regional development scenarios under various conditions and constraints. 
Traffic infrastructure plays a significant role in such developments; for example, very accessible areas are more attractive as residences and for commercial activities. 
Since accessibility is reduced by congestion, and congestion can only be realistically modeled through a sophisticated model of demand and supply interaction, \gls{urbansim} does not have its own travel model, but delegates that task to external models, such as \gls{matsim}.

To use MATSim4UrbanSim, one first needs to have a running \gls{urbansim} installation. 
From there, one can add \gls{matsim} to that installation; see %Section~\ref{sec:matsim4urbansim-stdInfo} 
the documentation mentioned above for more information. 
Basic \gls{matsim} parameters are configured from the \gls{urbansim} configuration file by adding an appropriate section; again, see 
%Section~\ref{sec:matsim4urbansim-stdInfo} 
the documentation mentioned above for more information. 
It is possible to add a standard \gls{matsim} \gls{configfile} allowing use of the extended \gls{matsim} features, including those added after the adapter package was designed.

The module was applied by \citet{CabritaEtcSustaincityHandbookBrusselsChapter} and by \citet[][]{ZoelligRenner_PhDThesis_2014}.
%% , but it has not been maintained since 2013 and it is unclear how much of it currently works.

% ##################################################################################################################
\section{NetworkEditor}
\label{sec:contrib-networkEditor}

\createStandardInformationBasic{{\url{http://matsim.org/javadoc} $\to$ networkEditor}}{%
%
\lstinline|RunNetworkEditor|%
%
}{%
%
none
%
}{%
%
none
%
}

This is, beyond the two network editors described in Chapter~\ref{ch:networkeditor}, a third network editor. 
It is older than the other two and has not been systematically maintained, but it still seems to be working and so it is still there.

% ##################################################################################################################

% Local Variables:
% mode: latex
% mode: reftex
% mode: visual-line
% TeX-master: "../../main"
% comment-padding: 1
% fill-column: 9999
% End: 
