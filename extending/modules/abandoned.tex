\chapter{Discontinued Modules}
\label{ch:discontinued}
% ##################################################################################################################

\hfill \textbf{Authors:} Kai Nagel, Andreas Horni

% ##################################################################################################################
This chapter lists modules that have been important in the past, but whose development is discontinued.

% ################################################################################################################
\section{DEQSim}
\label{sec:deqsim}
\emph{DEQSim} was used for project \emph{Westumfahrung} \citep[][]{BalmerEtAl_ResRep_bdktzrh_2009}. It was a queue-based, event-based parallel simulation written in C++ \citep[][]{CharyparEtAl_TRR_2007, Charypar_PhDThesis_2008}. This simulation included handling of reduced capacities due to traffic lights in an aggregate manner \citep[][p.139 ff]{Charypar_PhDThesis_2008}. It further supported modeling of gap back propagation at junctions \citep[][p.98 ff]{Charypar_PhDThesis_2008}. Events handling was done via file input and output. This represented a major bottleneck in terms of framework performance and thus it was replaced by a JAVA version, the \emph{JDEQSim} mobsim (see Chapter~\ref{???} \kai{Andreas, gibt es da ein Kapitel zu?}).

% ################################################################################################################
\section{Planomat}
\label{sec:planomat}
%\kai{Müssen wir das hier wirklich drinlassen?  Es hatte ja schon seinen Grund, dass wir das aus MATSim entfernt haben.}
%\ah{Ist die Frage, ob man die Historie noch mitschleppen will oder nur den aktuellen Stand beschreiben will. Bin in diesem Fall etwas unschlüssig. Planomat war für zwei grosse Projekte hier ein wichtiges Teil, deshalb wird man gute Argumente hören dafür es drinzulassen. Für den aktuellen Stand ist es aber tatsächlich nicht mehr relevant.}

Chapter~\ref{ch:extensionpoints} explains how MATSim can be extended.  One extension point that had been around for a long time is the \lstinline{PlanStrategy} extension point \ref{ch:extensionpoints}~\ref{sec:replanning-extension-point}.  It allows the addition of ``innovative'' strategy modules (cf.\ Section~\ref{sec:strategymodules}) beyond those available by default.

One such replanning model was Planomat \citep[][]{MeisterEtAl_IATBR_2006, MeisterEtAl_STRC_2006, Meister_PhDThesis_2011}.  It replaced the randomizing innovative modules  for (departure) time  (Section~\ref{sec:timechoice}) and for mode  (Section~\ref{sec:modechoice}) by a module that computed a joint best reply for both choice dimensions.  It did that by internally using a Genetic Algorithm, thus 
%% A special replanning module using a different logic than undirected trial-and-error was Planomat. Planomat did 
not just evaluating one random alternative per iteration as standard MATSim would do, but multiple alternatives within one single iteration, in order to obtain a (at least locally) optimal solution. 
%% It used a genetic algorithm  for this purpose. 
Planomat was successfully applied in the 
project \emph{KTI Frequencies} for time choice and mode choice for sub-tours \citep[][p.10]{BalmerEtAl_ResRep_datapuls_2010}.
Unfortunately, there were three interacting problem complexes with Planomat:
\begin{itemize}

\item Planomat did not re-use core MATSim structural information, but built its own model of the environment.  For example, it would not use the MATSim core scoring function provided by the infrastructure, but compute its own score based on the events. \kai{Thibaut or Andreas H., is this correct?} \ah{at least for the KTI frequencies project based on Planomat this is true.}  

This had the consequence that Planomat's idea of what was a good plan over time diverged from what core MATSim's idea of a good plan.  %% In other words, any change to the MATSim default scoring function would have to be re-implemented for Planomat.

\item Planomat had a tendency to always return the same solution.  While this is to be expected from a best-reply module, it becomes a problem when what the module thinks is a best reply is no longer consistent with what the core thinks.  

While an innovative strategy which deliberately generates  diversity can be useful even when not fully consistent with the MATSim core (cf.\cite{NagelKickhoeferJoubertHeterogeneousVoTs}), this is impossible for a non-diverse innovative strategy, since it insists on returning suboptimal plans and nothing else.

\item This went along with the fact that Planomat used the MATSim core router in a way that was not amenable to further software development of the core router.  Essentially, Planomat used MATSim classes and methods that were not designed for re-use, but just happened to be public.

It was thus in the way of a major re-design of the MATSim core router, undertaken by T.~Dubernet (cf.~Chapter~\ref{sec:routechoice} \kai{chk; link may make no sense}).

\end{itemize}

All three things together mean that Planomat was eventually abandoned: Moving it to the new router infrastructure would have meant a major piece of one-time work.  After that, maintaining Planomat's best-reply capability would have been a permanent work-intensive obligation.  It was thus decided to rather invest our scarce resources into the design of a better core that would allow for extensions to survive without a lot of manual intervention.  Although this will always be work in progress, Chapter~\ref{ch:extensionpoints} describes that we have moved forward in this direction quite a bit.

%% The Planomat module cannot be invoked directly anymore; it is not available in current releases anymore but can be downloaded from SVN history.
\kai{hier stand noch das mit der svn history.  ich vermute mal, dass das tatsächlich stimmt, aber wollen wir das wirklich nochmal ausprobieren?  Falls nicht, wollen wir das wirklich schreiben?}
\ah{Vermutlich sollte man die Leute tatsächlich "zwingen" erstmal mit dem MATSim-Team zu sprechen, anstatt sie implizit zu ermuntern auf eigene Faust sich das Zeug zu besorgen und loszulegen.}

% ################################################################################################################
\section{PlanomatX}

PlanomatX leaned on Planomat. It extended it by performing activity choice and adopting a Tabu Search approach \citep[][]{Feil_PhDThesis_2010}. To cope with curse of dimensionality due to the added choice dimension, PlanomatX introduced schedule recycling, which was basically a warmstart concept. Due to problems with a logarithmic utility function for activity choice, PlanomatX furthermore replaced to standard MATSim scoring function by an S-shaped function. Rough estimates for its parameters based on an multinomial logit model (MNL) exist. 

%% PlanomatX module cannot be invoked directly anymore; it is not available in current releases anymore but can be downloaded from SVN history.

PlanomatX, being derived from Planomat, suffered the same maintenance problems as Planomat.  It was thus eventually abandoned for the same reasons.

\ah{Könnte man hier (anstatt in Research Avenues) noch mehr auf die S-shaped function und ihre Probleme eingehen?
}

% ===================================================================================

%\kai{habe das queueSimulation Kapitel mal auskommentiert; falls nicht ok, sollten wir da nochmal drüber diskutieren.}
% \ah{ok}

%% \section{queueSimulation}
%% \label{sec:queueSimulation}

%% \kai{Anders als bei planomat finde ich, dass wir dies hier auch einfach komplett weglassen können.  Es gibt m.E.\ keine Veröffentlichungen, die nur Sinn machen, wenn man dies hier weiß; soweit wir es getestet haben, waren die beiden mobsims immer äquivalent.}

%% For some time, there existed two different versions of the QSim, the more advanced and experimental one called ``QSim'', and the more basic one named ``simulation'' or ``queueSimulation''.   Essentially, the basic version was forked off just before QSim exploded in terms of functionality and extensibility, in order to have a much simpler reference mobsim available.  

%% Over time, however, changes in the core made it necessary to adapt the basic queue simulation to those changes.  Many of these changes had to do with agent behavior, where the basic queue simulation was not able to differentiate, say, between activity end and departure.  In order to reduce the maintenance burden, eventually the agent concept in the basic queue simulation was retrofitted to the agent concept of the more advanced QSim.  In the end this meant that the basic queue simulation was slowly drifing away from the original fork, while on the other hand not being used and thus not tested by the MATSim core team.

%% It was thus eventually decided to remove the basic queue simulation from the repository, and concentrate on \lstinline{QSim}.
%% %% The queueSimulation was another mobsim forked from QSim. It cannot be invoked directly anymore; it is not available in current releases anymore but can be downloaded from SVN history. 
%% The \lstinline|simulation| config 
%% file section, which you might come across in old config files, is deprecated as well. 
%% %\ah{stimmt das so?}\michaz{Yes. IMO let's not even mention it. The fact that it was forked from QSim at all is a technicality which nobody wants to know.})


% ################################################################################################################
%\section{Discussions}
%% ===================================================================================
%\subsection{Planomat}
%\label{sec:planomat}
%\kai{Müssen wir das hier wirklich drinlassen?  Es hatte ja schon seinen Grund, dass wir das aus MATSim entfernt haben.}
%\ah{Ist die Frage, ob man die Historie noch mitschleppen will oder nur den aktuellen Stand beschreiben will. Bin in diesem Fall etwas unschlüssig. Planomat war für zwei grosse Projekte hier ein wichtiges Teil, deshalb wird man gute Argumente hören dafür es drinzulassen. Für den aktuellen Stand ist es aber tatsächlich nicht mehr relevant.}
%
%% ===================================================================================
%\subsection{PlanomatX ... Activity Choice}
%\label{sec:activitychoice}
%
%\kai{Müssen wir das hier wirklich drinlassen?  Es hatte ja schon seinen Grund, dass wir das aus MATSim entfernt haben.}
%\ah{Ähnlich oben, nur dass ich hier die Argumente dann nicht so gut nachvollziehen kann.}
%
%\kai{Ich habe das wie folgt in Erinnerung: (1) PlanomatX war mit heißer Nagel gestrickt.  Es gab nie ein Bemühem um code re-use, weder im code von Feil selber noch in bezug auf Matsim Infrastruktur.  Jedes "computational experiment" wurde definiert durch eine weitgehenden Kopie der gesamten code base.  Wir sahen uns nicht dazu in der Lage, hier dasjenige Subset zu extrahieren, welches einer sinnvollen Funktionalität entsprach.  (2) Durch die vielen Kopien war der playground beim refactoring massiv im Weg.  Es ist schon ok, wenn man bestimmte code Zeilen pro playground ein- oder zweimal austauschen muss, aber 14x ist zu viel.  (3) Michael Balmer sagte, dass er den Resultaten, insbesondere denen in den hinteren Kapiteln der Diss, nicht trauen würde (sonst hätten wir vielleicht ein paar test cases drumherum gebaut und versucht, es zu retten).}
%
%\kai{Ich sehe das von daher gesehen interessanterweise eher umgekehrt wie Du: Meiner Intuition nach hätte eher der normale planomat bleiben können.  Hier kann ich mich nicht dran erinnern, dass ich eine eigene Meinung hatte.  Als wesentliches technisches Problem habe ich in Erinnerung, dass planomat die Tendenz hatte, immer die gleiche Lösung zu finden, völlig unabhängig von den Präferenzen des Agenten (z.B.\ value of time).}
%
%\kai{Wenn es wg.\ der Historie ist, würde ich es in einer separaten Section unterbringen.}
%
%\kai{Falls es Euch wirklich wichtig ist, dann kann man ein ``contrib'' Kapitel ``planomat'' einfügen, analog zu den anderen contribs (accessibility, roadpricing, emissions, locationchoice, ...).  Wenn alles gut gelaufen wäre, dann wären auch planomat/planomatX jetzt contribs.  Bei PlanomatX wurde, wie gesagt, von Feil nie genug geliefert, dass das Chancen gehabt hätte.  Bei Planomat ist mir, wie gesagt, der Prozess nicht ganz klar.}
%
%\kai{Vermutlich brauchen wir irgendwann eine oder mehrere Skype Bereinigungstreffen, oder? :-)}
%
%\ah{Da habe ich mich missverständlich ausgedrückt: Die Argumente es \textbf{zu behalten} kann ich für PlanomatX nur schwer nachvollziehen, was mir bei Planomat besser gelingt.
%
%kwa ist aber auch mehr für "Current State" denn für "History". Allerdings wird wohl Planomat(X) von Balac wieder reaktiviert werden (müssen). D.h., irgendwo eine kurze Section zu "`Former Important Modules"' wäre evtl. nicht schlecht.}

% ===================================================================================
% ################################################################################################################

% Local Variables:
% mode: latex
% mode: reftex
% mode: visual-line
% TeX-master: "../../main"
% comment-padding: 1
% fill-column: 9999
% End: 
