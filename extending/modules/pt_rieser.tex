\section{Modeling Public Transport with MATSim \who{Rieser}}
%\hfill \textbf{Author:} Marcel Rieser

\ah{ to be deleted:
Allgemeiner Teil: Funktionalitäten und Besonderheiten

Modules in the config: 
\begin{itemize}
	\item \lstinline|transit|
	\item \lstinline|transitRouter|
	\item \lstinline|matrixBasedPtRouter|
	\item \lstinline|ptCounts|
\end{itemize}

Literature: \citet[][]{RieserNagel_IATBR_2009, Rieser_PhDThesis_2010}

http://matsim.org/docs/tutorials/transit

Standard Use-Case Berlin
}

\subsection{Introduction}

Public transport---or \emph{transit} as it is sometimes called---plays
an important role in many transport planning measures, even if they only target
non-transit modes in the first place. By making other modes more or less
attractive (e.g. by providing higher capacity with additional lanes, allowing
higher speeds, or charging money by setting up road pricing for an area), people
might reconsider their mode choice and switch to public transport (\emph{pt})
from other modes, or vice versa. Such changes can also occur when the transit
infrastructure is changed: additional bus lines, changed tram routes with
different stops served, altered headways, all have an influence whether
people tend to us a specific line, or public transport in general. Around 2007,
it was thus a central point to extend MATSim to support simulation of other
modes besides private car traffic, and especially to simulate public transport
in detail.

In a first step, MATSim was extended in such a way that modes other than
{\tt car} would be teleported, that is agents would be removed at one location
and placed at a later point of time---corresponding to the estimated travel time---at
their destination location, where they could commence their next activity.
Together with a simple mode-choice module, randomly replacing all the
transport modes in all legs of a plan, and a simple travel time estimation for
modes different than {\tt car}, first case studies resulting in modal share
changes were performed using MATSim
\citep{RieserGretherNagel2008modeChoiceCalculations,
GretherEtAl2009SimpleModeChoiceIPL}. This teleportation mode is now available by
default in MATSim and still a very good fallback to get a multi-modal scenario
up and running with as few data as possible.

In a second step, \gls{qsim} was extended to support the detailed simulation of
public transport vehicles that serve stops along fixed routes with a given schedule
\citep{Rieser2010}.
The next Section will describe the required data and resulting features for this
detailed public transport simulation in more detail.



\subsection{Data Model and Simulation Features}

MATSim supports modeling public transport in a high level of detail: transit
vehicles run along the defined routes of transit lines, picking up
and dropping off passengers at stop locations, while taking care of the transit
vehicles' capacities and maximum speeds. The data used to simulate public
transport in MATSim can be split in three parts:

\begin{compactitem}
\item Stop locations
\item Schedule, defining lines, routes and departures
\item Vehicles
\end{compactitem}

This data is stored in two files, with the vehicles being defined in one
file and stop locations and schedule being defined in
another file.
Examples of such files can be seen in
Section~\ref{sec:inputdata:transitvehicles} and
Section~\ref{sec:inputdata:transitschedule}, respectively.

The data model is comparable to that of other public transport planning
software, but simplified in a few aspects. A line has typically two or more
routes: One route for each direction, and additional routes when vehicles start
(or end) their service somewhere in the middle of the full route when coming
from (or going to) a depot. Each transit route contains a network route,
specifying along which links in the network the transit vehicle drives along,
and a list of departures, providing the information at what time a vehicle
starts at the first stop of the route. A route also describes the ordered list
of stops that are being served along with timing information, when a vehicle
arrives or leaves a stop. This timing information is given as offsets only, to
be added to the departure time at the first stop. Each departure contains the
time at which a vehicle starts the route, and a reference to the vehicle that
runs this service. Because the timing information is part of the route, routes
with the same sequence of stops may exist, differing only in the time offsets.
This is often the case with bus lines that take traffic congestion and
longer waiting times at stops during rush hours into account in the schedule. 

Stop locations are described by their coordinates and an optional name, and must
be assigned to exactly one line of the network for the simulation. Thus, they
can be best compared to ``stop points'' in VISUM. There is currently no logical
grouping of such stop locations to build a ``stop area''---a collection of
stops typically having the same name but e.g. on different arms of an intersection and
being served by different lines and where passengers can usually transfer
between).

Each vehicle belongs to one vehicle type. Such a vehicle type describes various
characteristics of a vehicle, like the seating and standing capacity (number of
passengers), its maximum speed, but also how many passengers can board or alight
a vehicle per second.

This data model already supports several advanced aspects of modeling public
transport, like having different travel speeds along routes during different
times of day (important for improved realism in the simulation), using vehicles
of different types on a route at different times of day (interesting for the
economic analysis of a schedule), and re-using transit vehicles for multiple
headways along one or different routes (allows the optimization of vehicle
deployment planning, or research the effects of delay-propagation).

With all this data set, the \gls{qsim} will simulate the movements of all
transit vehicles. The vehicles will start at the first stop of their route at the given
departure time, allow passengers to enter, and then drive along their route,
serving stops. At each stop, passengers can enter or leave the vehicle.
The simulation generates additional, transit-related events whenever a transit
vehicle arrives or departs at a stop, when passengers enter or leave a vehicle,
but also when a passenger cannot board a vehicle because the vehicle's capacity
limit is already reached. This allows for detailed analyses of the public transport
simulation performed by MATSim.

In order for passengers to use public transport in MATSim, they need to be able
to calculate a route using transit services. For this, MATSim includes a public
transport router which calculates the best route for arriving at the desired
destination with minimal cost, given a departure time. Costs are typically
defined as travel time only and a small penalty for changing lines, but other,
more complex cost functions could be used.

The routing algorithm is based on Dijkstra's shortest path algorithm
\citep{Dijkstra1959ShortestPath}, but modified in order to take
multiple possible transit stops around the start and the end coordinate
into account to find a route. Multiple start and end stops must be considered in
order to generate more realistic transit routes, as otherwise agents could
be forced to first travel into the wrong direction, or wait at an infrequently
served bus stop instead of going a bit further to a busy subway stop location.
By modifying the shortest path algorithm to work with multiple start and end
locations, a considerable performance gain was possible compared to the naive
implementation, calculating a route for each combination of start/end location
and then choosing the best out of it.


\subsection{Possible Improvements}

While the ability to simulate public transport was a big advance for MATSim,
there are still several shortcomings that could see improvement:

\begin{compactitem}
	\item The data model (and thus, the simulation) does not yet fully support some
	transit lines observed in the real world. Especially, circular lines without a
	defined start and end cannot be easily modeled yet. Also, some bus or
	train lines have stops where only boarding or alighting the vehicle is allowed,
	but not both (e.g. overnight trains with sleeper cabins). At the moment, MATSim
	always allows boarding and alighting at stops, leading to agents e.g. using a
	train with sleeper cabins only for a short trip, where in reality they would be
	denied boarding without a reservation for a longer trip.
	\item A stop location as seen by passengers in the real world is
	typically modeled as a number of stop facilities in MATSim, detailing the
	different locations where transit vehicles stop depending on their route and
	direction. For analysis purposes, one is often interested in aggregated values
	for such logical stop locations, and not for the individual stop facilities.
	Such a logical grouping is currently still missing in MATSim data format.
	\item Running simulations with a reduced sample of the population leads to
	artifacts when public transport is used. In a simulation with a sampled demand,
	network capacity is reduced accordingly to accommodate the fact that fewer
	private cars are on the road. But as still 100\% of public transport vehicles
	must run (albeit with reduced passenger capacity), the calibration becomes
	difficult. This should be solved in the future by not reducing the network
	capacity, but by giving each vehicle and agent a weight for how much it should
	count.
	\item The public transport router available and used by MATSim by default is
	strictly schedule based. It assumes, that the vehicles can keep up with the
	schedule and that enough passenger capacity is provided. In some regions, where
	transit offerings are notoriously delayed and overcrowded, MATSim's router will
	consistently advise agents to use routes that will perform badly in the
	simulation. Additional feedback from the simulation back to the router, as it
	is already done in the private car router of MATSim, will be needed.
	\item Last but not least, the current router based on a modified shortest path
	algorithm of Dijkstra can become rather slow and memory-intensive for
	larger areas with exhaustive transit offerings. Improved algorithms to generate
	the routing graph, or different routing algorithms altogether (like the
	non-graph based Connection Scan Algorithm
	\citep{DibbeltEtAl_BonifaciEtAl_2013}) will have to be researched in the
	future.
\end{compactitem}



\subsection{Applications}

The public transport simulation has been used in a variety of applications of
MATSim world-wide. The following list highlights some of these applications,
pointing out specialties of them regarding the public transport simulation.

\begin{compactitem}
	\item Berlin: The Berlin scenario (see Section~\ref{sec:berlinI}) was one of
	the first real applications using the public transport simulation in MATSim.
	The road and rail network as well as the full transit schedule was converted
	from a VISUM model. It is yet one of the few known models where bus and tram
	lines share a common network with private car traffic, enabling full
	interaction between private and public vehicles like transit vehicles getting
	stuck in and delayed due to a traffic jam.
	\item Switzerland: Senozon maintains a model of Switzerland, which contains the
	full time table of all buses, trams, trains, ships, and even cable cars in the
	Swiss alps. The schedule data is retrieved from the official time table,
	available in a machine-readable format called ``HAFAS raw data format''.
	\item Singapore: The model of Singapore (see Section~\ref{sec:singapore}) makes
	heavy use of public transport, and continually pushes the boundaries of what is
	currently possible to simulate. Due to the very large number of buses on
	Singapore's roads and the strong demand for it, a number of extensions had to
	be implemented in order to be able to realistically model pt in this context.
	\item Minibus: The minibus contribution (see next Section,
	\ref{sec:paratransit}) adds an optimization layer on top of the public
	transport functionality in MATSim, providing the functionality to automatically
	generate an optimized transit schedule for a certain region.
	\item wagonsim: In the wagonsim contribution (see Section~\ref{sec:wagonsim}),
	the public transport simulation was actually misused to simulate
	rail-bound freight traffic. While the simulation was still moving around
	transit vehicles and let passengers enter and leave such vehicles, the scenario
	was built in such a way that vehicles correspond to freight trains, and
	passengers correspond to the actual goods needing to be transported. Custom
	implementations of the transit driver logic replaces the definition of a
	vehicle's capacity by one that ensures that the trains the vehicles represent
	do not get too long or too heavy. The network was built in a way that changing
	vehicles at stops takes a minimum amount of time, corresponding to the time
	needed for switching wagons at freight terminals.
\end{compactitem}

Besides the applications mentioned in the list above, many additional scenarios
make use of the public transport simulation in MATSim by now. But the list also
shows, that with some custom extensions and imagination, the public transport
functionality can not only be used to ``just simulate public transport'', but be
used to solve complex problems previously being handled by operation research
groups.


