% ##################################################################################################################
\chapter{Analysis}
\label{sec:contrib-analysis}

\hfill \textbf{Author:} Kai Nagel

%% \subsection{Basic Information}

% Hallo Andreas, ich mag keine Überschriften, bei denen es nur ein X.X.1 gibt, aber kein X.X.2.  Also machen wir entweder "Basic Information" _und" "Summary" pro kurzem Eintrag, oder beides nicht.  Ich tendiere zu "beides nicht".  kai, may'15

\createStandardInformationBasic{%
%
\entryStd{analysis}
%
}{%
%
No standard invocation. See \url{http://matsim.org/javadoc} $\to$ analysis $\to$ \lstinline|RunKNEventsAnalyzer| class for intuition.
%
}{%
%
\configNonstd
%
}{%
%
\pubNone
%
}

%% \lstinline{org.matsim.contribs.analysis.*}

This \gls{contribution} collects various analysis tools for \gls{matsim} output.  

One important reason for having this in a \gls{contribution} rather than in a playground is the \gls{maven} layout of the repository: \Glspl{contribution} can use material from other \glspl{contribution}, but not from the playgrounds. 
In consequence, analysis tools that are needed in a \gls{contribution} need to be in a \gls{contribution} themselves. 
The analysis \gls{contribution} is a possible place where to put them.
