% ##################################################################################################################
\chapter{Matrix based pt router}
\label{sec:matrix-based-pt-router}

\hfill \textbf{Author:} Kai Nagel

%\ah{why not createStandardInfo?}\kai{As of now, createStandardInfo generates a ``section'' heading.  Cannot be used at subsection level.}

%% \subsection{Basic Information}

\createStandardInformationBasic{%
%
\entryStd{matrixbasedptrouter}%
%
}{%
%
\invokeStd{matrixbasedptrouter}{\lstinline|RunMatrixBasedPTRouterExample| class}%
%
}{%
%
\configStd{matrixBasedPtRouter}%
%
}{%
%
Section~3.1 of \citet{NicolaiNagelSustaincityHandbookMATSim4UrbanSim}; \citet{RoederNagel2013SketchPlanningBrussels}
%
}

The matrix based \gls{pt} router reads a list of \gls{pt} stops, and constructs ``teleported'' \gls{pt} routes using the stops nearest to origin and destination.  That is, each resulting trip will approximately look as follows:
\begin{xml}
<act type="previous" ... />
<!-- begin trip -->
<leg mode="walk" ... />
<act type="ptInteraction" ... />
<leg mode="pt" ... />
<act type="ptInteraction" ... />
<leg mode="walk" ... />
<!-- end trip -->
<act type="next" ... />  
\end{xml}

The attributes of the walk and the \gls{pt} legs will be computed from the coordinates of the locations in the same way as teleportation routing (see Section~\ref{sec:teleportation-routing}), and then taken at face value in the \gls{mobsim} (see Section~\ref{sec:teleportation-qsim}).

Travel times and travel distances between \gls{pt} stops can alternatively be given by corresponding matrices.  This is particularly useful if a \gls{pt} assignment exists and such information can be extracted from that.  This was used by \citet{RoederNagel2013SketchPlanningBrussels} and by \citet{ZoelligRenner_PhDThesis_2014}.

