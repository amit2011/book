% ##################################################################################################################
%% \section{Mobility Simulations}
%% \label{sec:extending-mobsims}
% ===================================================================================
\chapter{QSim}
\label{sec:extending-qsim}

\subsubsection{Vehicle Types and Vehicles}
\label{sec:vehicles-in-qsim}

For a variety of reasons---\eg vehicle-specific emissions calculations (Chapter~\ref{ch:emissions}), or vehicle-specific maximum speeds (see below)---it may become necessary to assign different vehicle types to different persons, modes, or trips.
%
The (arguably) most ``honest'' approach to vehicles, in terms of micro-simulation, is to generate a synthetic vehicle fleet.  Each leg would then have to know which vehicle it wants to use.  This is indeed possible with the planned vehicle \gls{id} that \Gls{matsim} route objects can store.  This functionality is switched on by first loading an additional vehicles file (see Sec.~\ref{sec:extending-vehicles}) and then configuring the \gls{qsim} as
\begin{xml}
<module name="qsim">
   <param name="usePersonIdForMissingVehicleId" value="false" />
   <param name="vehiclesSource" value="fromVehiclesData" />
   ...
</module>
\end{xml}
(available with release 0.8.x).  This states that every time the \gls{qsim} needs a vehicle with a specific \gls{id}, it will search for it in the vehicles data container, throwing an exception if it is not found there.  

\paragraph{A Fallback for Routes that do not Contain Vehicle \glspl{id}}

In the above approach, vehicular routes need to provide vehicle \glspl{id}, otherwise the \gls{qsim} will throw an exception.  Since algorithms to compute and maintain vehicle \glspl{id} during replanning are currently not well developed within \gls{matsim}, an alternative is to assume that persons use a vehicle with the same \gls{id} as the person.  This is switched on by
\begin{xml}
<module name="qsim">
   <param name="usePersonIdForMissingVehicleId" value="true" />
   ...
</module>
\end{xml}
which is also the current default.
%
With this configuration, it will still search for the vehicle \gls{id} in the route, but if this is unavailable, it will instead use the person \gls{id} as vehicle \gls{id}.  Without additional configuration, it will then still search for the vehicle under that \gls{id} in the vehicles file.

\paragraph{Alternative Vehicles Sources}

A default vehicles source is defined by
\begin{xml}
<module name="qsim">
   <param name="vehiclesSource" value="defaultVehicle" />
   ...
</module>
\end{xml}
This generates a default vehicle (typically a medium-sized 4-seater) every time a vehicle is needed and is currently the default configuration.

At the moment, alternative approaches to vehicle generation need to be programmed as script-in-\gls{java}.  
%% As an alternative, one can use \lstinline|VehicleUtils| for specifying mode vehicles according to the example syntax in 
See, e.g., 
\lstinline{RunMobsimWithMultipleModeVehiclesExample} under \url{http://matsim.org/javadoc} $\to$ main distribution
 %% (see also Section~\ref{sec:using-qsim-multimodal}). 
for a reference to a script that generates mode-specific typical vehicles for each mode, and Section~\ref{sec:extending-qsim} for a general explanation of scripts-in-\gls{java}.
Simulation experiments 
%for \gls{qsim}'s relatively new \gls{multimodal} feature, applying multiple mode vehicles, 
using this feature have been performed for the Patna scenario as reported in Chapter~\ref{ch:patna}.


\paragraph{Vehicle Behavior}

Vehicles need to be available where they are needed.  It is, for example, impossible to perform a trip by car, then another (non-circular) trip by public transit and then make another trip with the same car as before, since the car will not be available at that location.  The \gls{qsim} is able to enforce such behavior, with the setting
\begin{xml}
<module name="qsim">
   <param name="vehicleBehavior" value="exception" />
   ...
</module>
\end{xml}
This means that if a necessary vehicle is not available at the location where it is needed by the traveler, the \gls{qsim} throws an exception and aborts.  The idea here is that such synthetic travelers should have within-day replanning strategies (see Section~\ref{ch:withinday}) to cope with unexpectedly unavailable vehicles; any attempt to use an unavailable vehicle points to an error in the driver's behavioral logic.

In many standard situations, the above behavior will be too strict.  For example, a vehicle may be shared between family members, but one member will be late in returning a vehicle.  For such situations, 
\begin{xml}
<module name="qsim">
   <param name="vehicleBehavior" value="wait" />
   ...
</module>
\end{xml}
may be an option.  Here, a driver will wait if a vehicle is not available.  Errors in the coordination logic, i.e.\ very long waits, will be punished via the \gls{matsim} scoring logic,  thus leading to more robust coordinations.

A final alternative is
\begin{xml}
<module name="qsim">
   <param name="vehicleBehavior" value="teleport" />
   ...
</module>
\end{xml}
With this setting, vehicles will be teleported to locations where they are needed.

\paragraph{Initial Vehicle Placement}

For vehicle behavior of type \lstinline{exception} and type \lstinline{wait}, vehicles need to be at the correct location when the \gls{qsim} starts.  Here, the default simulation currently places all vehicles at the home location -- for other variants, some additional code needs to be written or used, such as the car sharing extension (Chapter~\ref{ch:carsharing}).

%% Here, vehicles may not always be at the home location at simulation start.  For example, some people keep bicycles at the arrival location of their morning train commute for leg between train station and work location.  In other situations, a company vehicle may have to be picked up before it can be used.

%% At this point, the \gls{qsim} locates \kaitodo{???}





%% % ..............................
%% \paragraph{Different Vehicle Types by Mode}
%% %The \lstinline|PassingQ| makes no difference, as long as all vehicles have the same maximum speed. 
%% %It is thus possible to define so-called ``mode vehicles'', \ie a typical vehicle for each mode. 
%% %\ah{was not easy to read. reformulated it. pls check}\kai{ok}
%% The option to simulate scenarios with 
%% %\st{equal} different \kai{Andreas, ok?} 
%% %% \ah{Wollte eigentlich sagen: Identisch pro Mode. mode-specific müsste reichen oder?}
%% %% \kai{Klingt komisch ohne das ``maximum'', habe es daher dringelassen.}
%% mode-specific maximum speeds is supported by the possibility to define so-called ``mode vehicles'', \ie by specifying a typical vehicle for each mode.
%% %
%% %% With the two options \lstinline{PassingQ} and \lstinline{FIFO}, passing is implemented as all-or-none, \ie either every vehicle type can pass every other, or none can pass. More fine-grained passing rules are currently being researched.
%% %This is discussed \kai{where?}.
%% %\footnote{%
%% %  
%% %\ah{ab hier vielleicht nach Part II + forward pointer?  Dann haben wir wieder Konsistenz.}
%% %
%% %\kai{Ja das koennen wir versuchen.} \ah{ok}
%% %
%% %In order to not use the pre-defined default vehicle type for each mode, for the time being it is necessary to write a script-in-\gls{java}. \ah{moved this down}
%% %For the necessary syntax, see the \lstinline{RunMobsimWithMultipleModeVehiclesExample} under \url{http://matsim.org/javadoc} $\to$ core.  
%% %We will eventually make this available as a config option; at this time, we are still searching for a software design that makes the QSim both pluggable at multiple levels \emph{and} easy to maintain.
%% %
%% %}
%% %
%% %% So as not to use the same pre-defined default vehicle type for each mode, 
%% For this functionality it is necessary, at the moment, to write a script-in-\gls{java}; see Section~\ref{sec:extending-vehicles} for a reference to the script and Section~\ref{sec:extending-qsim} for a general explanation of scripts-in-\gls{java}.\footnote{We will eventually make this available as a config option; at this time, we are still searching for a software design to make the \gls{qsim} both pluggable at multiple levels \emph{and} easy to maintain.}

%% %% \ah{Should we say something about this? Is that correct?}
%% %% \ah{Passing is all-or-none, \ie either every vehicle type can pass every other type or none can pass none. 
%% %% However, in reality, there usually is something between \gls{fifo} and free passing, \eg a bus might not be able to pass a car, although being faster, whereas bikes might always be able to slip by cars. Such more fine-grained rules about passing do not yet exist at the time of writing this book. }
%% %% \kai{Hm.  Ist nicht ganz einfach zu beschreiben, was da wirklich passiert.  Habe schon viele Stunden mit dem gemeinsamen Paper mit Amit et al verbracht, und endgueltig klar ist es immer noch nicht.  }
%% %% \ah{Die Frage wird m.E. von jedem Leser kommen. Also einfach?: Passing is for now implemented as all-or-none, \ie either every vehicle type can pass every other type or none can pass none. More fine-grained passing rules are currently researched.}
%% %% \kai{s.o.}

%% % ..............................
%% \paragraph{Planned Vehicle Id}
%% Routes, as described in Section~\ref{sec:lgstarted-population-file}, can have a planned vehicle \gls{id} as attribute (\lstinline|vehicleId|).  
%% At the same time, all available vehicles need to be described by an additional file. 
%% The \gls{qsim} will then attempt to use that specific vehicle, including its individual attributes such as maximum speed.  
%% This is discussed \kaitodo{where?} \ahtodo{where?}.

%% It is possible to give each planned route a list of possible vehicle \glspl{id} to be used for this route.  
%\ah{I think this is not easy to understand, but yes, the ref will probably clarify this}
%\kai{Re-wrote it.}

% ..............................
\paragraph{PassingQ} 

Once various vehicles have different maximum speeds, the standard \gls{qsim}, even with multiple main modes, is no longer sufficient, since it uses \gls{fifo} as the queuing discipline, meaning that fast vehicles cannot pass slower vehicles. Here, the so-called \lstinline|Passing(Vehicle)Q| can be used instead.  It replaces the \gls{fifo} sorting criterion---where vehicles are sorted by the sequence in which they arrive on the link---by a sorting employing the so-called earliest link exit time, computed from link enter time and freespeed travel time.  Now, using the minimum of vehicle and link maximum speeds, the freespeed travel time can be differentiated between vehicles, allowing fast vehicles to obtain an earlier link exit time, even if they enter later than slow vehicles.  Details and resulting fundamental diagrams are given by \citet{AgarwalEtcMixedTraffic}.

This option can be enabled by using
\begin{xml}
<module name="qsim">
   <param name="linkDynamics" value="passingQ" />
   ...
</module>
\end{xml}
in the \lstinline{qsim} section of the \gls{configfile}.


\subsubsection{Other}

The simulation is able to handle time-variant networks \citep{LaemmelGretherNagel2009TimeDependentNetworks}, within-day replanning \citep[][see Chapter~\ref{ch:withinday}]{Dobler_TechRep_IVT_2009} and traffic lights \citep[][see Chapter~\ref{ch:signalslanes}]{Neumann2008DA,GretherNeumannNagel2011TrafficLightsMatsim,GretherNeumannNagel2012SignalsQueueModelABMTrans}).
%
%% Section~\ref{sec:using-othermodesthancar} details the necessary steps to simulate a multi-modal scenario; Section~\ref{sec:using-qsim-multimodal} therein focuses on \gls{qsim}. 
%
An earlier \gls{multimodal} approach, targeted at overcoming the \gls{teleportation} estimates of non-motorized modes, and particularly focused on pedestrians, is presented in Chapter~\ref{ch:multimodalsim}.

% --------------------------------------
%\subsubsection{Multi-Modal Simulation With QSim}
%\label{sec:multimodalsim_qsim}
%\gls{qsim} can simulate \gls{multimodal} traffic. Congested modes, \ie modes simulated not only \gls{teleported}, are specified by the parameter \lstinline|mainMode|. Technically, these are the modes that the departure handler of the \lstinline|netsimengine| handles 
%\footnote{Effective cell size, effective lane width, flow capacity factor, and storage capacity factor need to be set with diligence.}. 
%Theoretically also pedestrians can be included, however, only vehicular modes actually make sense, \ie car, bike, and public transport, in particular buses.
 
%\ah{Is that true with pt? Do we actually have all the vehicular traffic interactions in \gls{qsim}?}. 
%Assigning vehicles of a specific type to agents is done with in the vehicles input file \lstinline|<param name="vehiclesFile" value="..." />|. 
%\ah{is that the only place necessary?}. 
%Further configuration necessary to include public transport is detailed in Chapter~\ref{ch:pt}.

%To model mode interactions, a fully \gls{multimodal} network, \ie a network that has links being shared by different modes must be provided. 
%As shown in Section~\ref{sec:lgstarted-network-file}, specification of link modes is done with the network attribute \lstinline|modes|.
%Furthermore, the router needs to be informed about the network modes by the \lstinline|planscalcroute| parameter \lstinline|<param name="networkModes" value="car,ride" />|.
%\ah{is this a double configuration, qsim and planscalcroute? can we merge them?}

%The mode interactions can be further specified by \gls{qsim} parameter \lstinline|linkDynamics|. 
%This parameter is also productive for car mode only, but it is particularly important for multi-modal scenarios. 
%Dynamics either follow strict \gls{fifo} or include passing, \ie \lstinline$<param name="linkDynamics" value="FIFO|PassingQ" />$. 

%Simulation experiments for \gls{qsim}'s relatively new \gls{multimodal} feature, applying multiple mode vehicles, have been performed for the Patna scenario as reported in Chapter~\ref{ch:patna}.

%An earlier \gls{multimodal} approach targeted at overcoming the \gls{teleportation} estimates of non-motorized modes, particularly focused on pedestrians, is presented in Chapter~\ref{ch:multimodalsim}.

%\ah{Amit, Kai: Please check this section. Thanks.}

