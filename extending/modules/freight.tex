\chapter{Freight Traffic}
\label{ch:freight}
% ##################################################################################################################

\hfill \textbf{Authors:} Michael Zilske, Johan W. Joubert

\ah{titlefigure? asked by mail}

% ##################################################################################################################
Different approaches to model freight traffic in MATSim have been implemented in recent years. 

For Zürich, available origin-destination matrices for small delivery trucks and heavy trucks have been disaggregated \citet[][]{ShahM_TechRep_IVT_2010}. Data was taken from the \emph{KVM} (Kantonales Verkehrsmodell) provided by \citet{AMV_Webpage_2011} and documented in \citet[][]{GottardiBuergler_SV_1999}. This special freight subpopulation then has been restricted to route choice.

In South Africa, the plans of freight vehicles were derived from the GPS records of more than 30\,000 commercial vehicles that were tracked over a 6-month period. The extraction of activity chains from the raw GPS data was documented in \citet[][]{JoubertAxhausen_JTG_2011}, while the first joint implementation of private cars and freight appeared in \citet[][]{JoubertJEtAl_TRR_2010}. In \citet[][]{NagelKickhoeferJoubert2014HeterogeneousVoTsPROCEDIA}, we used MATSim to evaluate the impact of a complex toll structure that is vehicle type specific and where the different subpopulations, including freight, have different values of time.

The most sophisticated solution however, was the introduction of carrier agents as described in the following section. 

% ##################################################################################################################
\section{Carriers}
\label{sec:carriers}
Up to now, real-world scenarios set up with MATSim have modeled the freight traffic share of the demand by using a set of plans with activities at the depot and at pick-up and delivery locations, but with no variability in any choice dimension except route choice. We improve on this situation by modeling freight vehicles as non-autonomous agents employed by and serving the interests of freight operators. The missing choice dimensions of freight vehicle drivers are then realized as logistics decisions made by the freight operators who employ them. In the freight transport sector, decisions are distributed among actors with different roles. Freight transport decisions include lot-size choice, path-searches in logistical networks, vehicle choice and tour planning. The planning problem of a freight operator is therefore quite different from its passenger counterpart:

Firstly, the success of freight transport plans is not determined by the utility of time
spent at activity locations, but rather by their commercial success. They have to meet the
requirements of customers, like meeting time windows and providing enough capacity at
a reasonable cost.

Secondly, freight operators often do not operate one single vehicle but several, and their
options include rescheduling deliveries from one vehicle to another or even changing the
number of vehicles which are used at all.

Therefore, a new software layer populated by \emph{carrier agents} was introduced into the
simulation. Each carrier agent represents a firm with a vehicle fleet, depots and contracts.
Contracts determine type and quantity of goods to be carried. A contract contains the respective 
origin and destination as well as pick-up and delivery time windows.

The plan of a carrier agent contains a schedule of a tour for each of the vehicles in the fleet. 
The schedule contains planned pick-up, delivery or arrival times at customer locations and a route through 
the physical network. In our basic model, all vehicle schedules of a carrier begin and end at one of its depots.
When a simulation scenario is initialized, the carrier agents build a schedule for each of their vehicles, 
including a route through the transport network, with pick-up and delivery activities corresponding to their contracts.
At the interface between the freight operator plans and the mobility simulation, the set of routed vehicles 
from each carrier plan is injected into the traffic demand as individual driver agents, where they move 
through the traffic system along with passenger vehicles. While executing their plans, the freight driver 
agents report their shipment-related activities back to the carrier.

When all plans have been executed, agents evaluate the success of their plan. The carrier agents use a custom 
utility function that captures their economic success. Their cost is calculated as a sum of vehicle-dependent 
distance and time costs incurred by their scheduled vehicles, and some individual fixed costs, plus penalties 
incurred by missed time windows.

Finally, carrier agents create new plans to improve their performance in the next iteration. For instance, a time dependent vehicle routing heuristic can be plugged in to replan vehicle schedules. Shipments can be switched between vehicles, or even an entire vehicle added or removed.
During repeated executions of their plans, passengers as well as carriers collect experience from the transport system. The carriers pick up congestion and other disturbances in the traffic system when they incur a higher cost through longer vehicle usage, or by penalizing missed pick-up and delivery times.

The planning algorithms themselves are implemented in the project \lstinline|jsprit|, a library separate from MATSim.
In the replanning phase of each iteration, jsprit is called and replans the carrier plans.

For more details about the implementation as well as more references see the technical report by \citet[][]{ZilskeEtAl_TechRep_VSP_2012}.

% Package that plugs freight algorithms (programmed in external package jsprit) into matsim.
% * A good starting point is <a href="https://github.com/jsprit/jsprit/wiki/Network-Based-VRPs"> 
% * https://github.com/jsprit/jsprit/wiki/Network-Based-VRPs</a>.

% ##################################################################################################################
