\chapter{Freight Traffic}
\label{ch:freight}
% ##################################################################################################################

\hfill \textbf{Authors:} Michael Zilske, Johan W. Joubert

% ##################################################################################################################


\section{Carriers}
\label{sec:carriers}

Up to now, real-world scenarios set up with MATSim have modeled the freight traffic share of
the demand by using a set of plans with activities at the depot and at pick-up and delivery
locations, but with no variability in any choice dimension except route choice. We improve on
this situation by modeling freight vehicles as non-autonomous agents employed by and serving
the interests of freight operators. The missing choice dimensions of freight vehicle drivers are
then realized as logistics decisions made by the freight operators who employ them. In the
freight transport sector, decisions are distributed among actors with different roles. Freight
transport decisions include lot-size choice, path-searches in logistical networks, vehicle choice
and tour planning. The planning problem of a freight operator is therefore quite different from
its passenger counterpart:

Firstly, the success of freight transport plans is not determined by the utility of time
spent at activity locations, but rather by their commercial success. They have to meet the
requirements of customers, like meeting time windows and providing enough capacity at
a reasonable cost.

Secondly, freight operators often do not operate one single vehicle but several, and their
options include rescheduling deliveries from one vehicle to another or even changing the
number of vehicles which are used at all.

Therefore, a new software layer populated by \emph{carrier agents} was introduced into the
simulation. Each carrier agent represents a firm with a vehicle fleet, depots and contracts.
Contracts determine type and quantity of goods to be carried. A contract contains the respective 
origin and destination as well as pick-up and delivery time windows.
The plan of a carrier agent contains a schedule of a tour for each of the vehicles in the fleet. 
The schedule contains planned pick-up, delivery or arrival times at customer locations and a route through 
the physical network. In our basic model, all vehicle schedules of a carrier begin and end at one of its depots.
When a simulation scenario is initialized, the carrier agents build a schedule for each of their vehicles, 
including a route through the transport network, with pick-up and delivery activities corresponding to their contracts.
At the interface between the freight operator plans and the mobility simulation, the set of routed vehicles 
from each carrier plan is injected into the traffic demand as individual driver agents, where they move 
through the traffic system along with passenger vehicles. While executing their plans, the freight driver 
agents report their shipment-related activities back to the carrier.
When all plans have been executed, agents evaluate the success of their plan. The carrier agents use a custom 
utility function that captures their economic success. Their cost is calculated as a sum of vehicle-dependent 
distance and time costs incurred by their scheduled vehicles, and some individual fixed costs, plus penalties 
incurred by missed time windows.
Finally, carrier agents create new plans to improve their performance in the next iteration. For instance, 
a time dependent vehicle routing heuristic can be plugged in to replan vehicle schedules. Shipments can be 
switched between vehicles, or even an entire vehicle added or removed.
During repeated executions of their plans, passengers as well as carriers collect experience from the transport 
system. The carriers pick up congestion and other disturbances in the traffic system when they incur a higher 
cost through longer vehicle usage, or by penalizing missed pick-up and delivery times.

The planning algorithms themselves are implemented in the project jsprit, a library separate from MATSim.
In the replanning phase of each iteration, jsprit is called and replans the carrier plans.

For more details about the implementation as well as more references see the technical report by \citet[][]{ZilskeEtAl_TechRep_VSP_2012}.


% Package that plugs freight algorithms (programmed in external package jsprit) into matsim.
% * A good starting point is <a href="https://github.com/jsprit/jsprit/wiki/Network-Based-VRPs"> 
% * https://github.com/jsprit/jsprit/wiki/Network-Based-VRPs</a>.


% ##################################################################################################################
%\section{Zurich}
%
%Literature: \citet[][]{ShahM_TechRep_IVT_2010}
%
%%Sebastian Schneider has done a Ph.D. dissertation about generating freight vehicle chains by essentially re-sampling the information
%% * contained in the German survey KiD.  Since the KiD is essentially an activity-based travel diary, the method should also be applicable
%% * to other situations.
%% * <p/>
%% * Since Sebastian has left science for the time being, he allowed me to take his code and integrate it into the matsim repository, under
%% * the GPL.  For the time being, it will just "sit" here until someone attempts to make it work.
%% * 
%% * @author (of this documentation) Kai Nagel
%% * 
%% * @author (of code) Sebastian Schneider
%
%Canton Zurich raw data for freight traffic is taken from the \emph{KVM} (Kantonales Verkehrsmodell) provided by \citet{AMV_Webpage_2011} and documented in \citet[][]{GottardiBuergler_SV_1999}. Matrices given at zonal level (1341 zones, 182'687 trips) are disaggregated to single MATSim plans \citep[][]{ShahM_TechRep_IVT_2010}. Matrices for small delivery trucks and heavy trucks are combined into one activity called \emph{freight}. An additional 181'886 agents are generated for Zurich region. 
%
%Base data from the \emph{KVM} includes OD-matrices for three specific hours of the day and the daily volumes. These matrices have been manually converted from Visum format to text files.
%
%It is not fully clear to the authors if the Swiss microcensus captures some of the freight traffic through the activity type \emph{"Geschäftliche Tätigkeiten und Dienstfahrt"}. However, only $6\%$ of all trips are done for this purpose, thus, the quantitative error is expected to be small in any case.

% ##################################################################################################################
\section{South Africa}
In South Africa, the plans of freight vehicles were derived from the GPS records of more than 30\,000 commercial vehicles  that were tracked over a 6-month period. The extraction of activity chains from the raw GPS data was documented in \citet[][]{JoubertAxhausen_JTG_2011}, while the first joint implementation of private cars and freight appeared in \citet[][]{JoubertJEtAl_TRR_2010}. In \citet[][]{NagelKickhoeferJoubert2014HeterogeneousVoTsPROCEDIA} we used MATSim to evaluate the impact of a complex toll structure that is vehicle type specific and where the different subpopulations, including freight, have different values of time.

% ##################################################################################################################
