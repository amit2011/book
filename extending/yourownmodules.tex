\chapter{How to Write Your Own Modules and Contribute Them to MATSim}
\label{ch:extensionpoints}
% ##################################################################################################################
\hfill \textbf{Authors:} Michael Zilske

% ##################################################################################################################
\section{Introduction}
\label{sec:ownmodules-intro}

\subsection{Extensions to and modifications of the co-evolutionary process}

As explained in Section~\ref{sec:co-ev}, \acrshort{matsim} at its core is a co-evolutionary cycle consisting of the three elements \verb$Mobsim$, \verb$Scoring$, and replanning.  All three elements operate on what is essentially an in-memory object-oriented data base of \verb$Person$s \citep{RaneyNagel2006traf-framework}.  
%
On a conceptual level, each \verb$Person$ object is a Q-learning agent whose action consists of selecting a \verb$Plan$.  The plan is then executed in the \verb$Mobsim$; the executed plan obtains a score in the \verb$Scoring$; plans are possibly mutated and finally selected in \verb$Replanning$.

In consequence, it is these three elements which can be reconfigured:
\begin{compactitem}

\item The \textbf{\tt Mobsim} can be replaced, either by an internally available alternative, or by a fully external \verb$Mobsim$.

%% In addition, one can insert additional instructions into the program flow of \verb$QSim$, which is the default implementation of the \verb$Mobsim$.  This is achieved by so-called \verb$MobstimListener$s.

\item The \textbf{\tt Scoring} can be replaced, by possibly giving each individual agent a different recipe to compute its score.

\item Arbitrary implementations of type \verb$PlanStrategy$ can be added to the replanning -- these either generate new plans from scratch or by mutating existing ones, or they select between plans.

\end{compactitem}
It is also possible to insert additional material between these elements.  This is achieved by so-called \verb$ControlerListener$s.

The \verb$Mobsim$ generates a stream of \verb$Event$s.  These are primarily used in two places:
\begin{compactitem}

\item The \verb$Scoring$ uses them to track each agent's success at executing its plan, and compute the scoring value based on this.

\item \verb$PlanStrategy$ modules use them to build approximate models of the world in which they operate.  For example, the router obtains time-dependent expected link travel times from a \verb$TravelTime$ object, which in turn listens to link enter and link exit events.
  
\end{compactitem}
In consequence, each \verb$PlanStrategy$ module can use them to obtain information about the world in which the plans of the synthetic travelers need to be improved. Additionally, one can write nearly arbitrary handlers for analysis: If one wanted to and if the corresponding information is in the data, then one could extract the average age of all drivers using roads with speed limits 30 or less.


\subsection{Extensions to and modifications of specific elements}

Some modules are so large that fully replacing them in order to adapt the simulation system to one's need is too much work.  These are, in particular,
\begin{compactitem}
\item the \verb$QSim$, which is the default implementation of the \verb$Mobsim$, and
\item the router.
\end{compactitem}
In consequence, it is possible to add material into the execution flow of the \verb$QSim$ by \verb$MobsimListeners$ in a similar way as this is possible with the \verb$ControlerListeners$ mentioned above.
%
The router, in contrast, is most importantly configured by replacing the definition of the generalized travel cost. 

% ##################################################################################################################
\section{MATSim Extension Points}
This section describes what could be called the service provider interface (SPI) of MATSim.
Historically, the main entry-point for writing a MATSim extension has been to literally extend (in the Java sense)
the Controler class.  Essentially, one would override the methods calling the mobsim, the scoring, and/or the replanning, as explained in Section~\ref{sec:ownmodules-intro}. This is now discouraged. While this pattern worked when a each member of the team
was working on extending the MATSim core by a different aspect, it fails when it comes to
integrating those aspects to a single product: There is nothing one can do with a PublicTransportControler, an EmissionsControler, a RoadPricingControler and an OTFVisControler,
if one wants to combine them to visualize the emissions of buses on toll roads.

% =========================================================================================
\subsection{Config Group}
\label{sec:config}
The configuration of a MATSim run is a grouped list of key-value pairs, stored in XML
format, often in a file named \verb$config.xml$:
\begin{lstlisting}
<module name="qsim" >
		<param name="flowCapacityFactor" value="0.2" />
		<param name="storageCapacityFactor" value="0.3" />

		<!-- `queue' for the standard queue model, `withHolesExperimental' (experimental!!) for the queue model with holes -->
		<param name="trafficDynamics" value="queue" />
</module>
<module name="transit" >
		<!-- Comma-separated list of transportation modes that are handled as transit. Defaults to 'pt,bus,train,tram'. -->
		<param name="transitModes" value="pt" />

		<!-- Input file containing the transit schedule to be simulated. -->
		<param name="transitScheduleFile" value="network/transitSchedule.xml.gz" />
</module>
\end{lstlisting}
At runtime, the entire configuration is stored in an instance of \verb$Config$, from which instances of \verb$Module$ can be accessed by their name.
The author of an extension can subclass the \verb$Module$ class to provide named accessors for the parameters.
There is even a utility class \verb$ReflectiveModule$ which you can use if you want to define the mapping of named parameters to accessors
using Java annotations.

% =========================================================================================
\subsection{ScenarioElement}

As explained in Section~\ref{sec:ownmodules-intro}, the \verb$Person$s are stored in an in-memory object-oriented data base.  This means that they are read into memory at the beginning of a run, kept in memory during the whole run, and written to files at the end of a run.
%
The same approach, i.e.\ reading a file into memory and keeping it there, has always been used for the network, consisting of links and nodes, although here it is less normal that this data is modified over the iterations.

Since then, additional data containers were introduced for vehicles, households, the transit schedule, and activity facilities (see Section \kai{which one}).  These provide additional modelling options, yet they are only loosely connected to the \acrshort{matsim} core.  It then became increasingly obvious that there is no standard set of data containers, but different modelling tasks need different containers.  For example, traffic signals need signal plans, while freight modelling needs a separate population of freight forwarders.  It was thus decided to allow for the addition of arbitrary scenario elements, by the syntax
\begin{lstlisting}
  scenario.addScenarioElement( "myContainer", new MyContainer() ) ;
\end{lstlisting}
This is then retrieved by
\begin{lstlisting}
  (MyContainer) scenario.getScenarioElement( "myContainer" ) ;
\end{lstlisting}

\subsection{Controler}
\label{sec:controlerextension}
%http://ci.matsim.org:8080/job/MATSim_M2/javadoc/org/matsim/core/controler/package-summary.html
Controler remains the main user-facing class of MATSim, but please do not subclass it. Rather,
use its setter methods to plug in your own code.

% =========================================================================================
\subsection{ControlerListener}

ControlerListeners are called at the transitions of the washing-machine diagram, in undefined order.
Thus, AControlerListener may only rely on the computation of BControlerListener if BControlerListener
 makes that computation in an earlier transition. For instance, if BControlerListener is a StartupListener
 and loads data into a Map on start-up, AControlerListener can be an IterationStartsListener and use that Map.
 But do not write two IterationStartsListeners where the first puts some data into a Map and the second expects
 to find it there - they may be called in any order.

% =========================================================================================
\subsection{Events}
\label{sec:events}
\subsubsection{What they are}
The mobility simulation moves the agents around in the virtual world according to their plans and within the bounds of the ``simulated reality''. The mobility simulation documents its moves with so-called ``Events''. These events are small pieces of information describing the action of an object at a specific time. Examples of such events can be:

    An agent finishes an activity
    An agent starts a trip
    A vehicle enters a road segment
    A vehicle leaves a road segment
    An agent boards a public transport vehicle
    An agent arrives at a location
    An agent starts an activity

Each event has a timestamp, a type, and additional attributes required to describe the action like the agent's id, a link id, an activity type or other data. In theory, it should be possible to replay the mobility simulation just by the information stored in the events. While plans describe the agents' plan for a day, the events describe how the agents' day actually was (according to the simulation).

As the events are so basic, each agent typically generates hundreds of events during one execution of a mobility simulation. In total, the number of events generated by a mobility simulation can easily reach a million or more, with large simulations even generating more than a billion events. But as the events really describe all the details from the execution of the plans, it is possible to extract mostly any kind of aggregated data one is interested in. Practically all analyses of MATSim simulations make use of events to calculate some data. Examples of such analyses are the average duration of an activity, average trip duration or distance, mode shares per time window, number of passengers in specific transit lines and many more.

The scoring of the executed plans makes use of events to find out how much time agents spent at activities or for traveling. Some replanning modules might make use of events as well: The router for example can use the information contained in events to figure out what links are jammed at certain times and route agents around that jam when creating new plans.
``

\subsubsection{How to handle them}
MATSim extensions can watch the mobility simulation by interpreting the stream of Events. This is done
by implementing an EventHandler interface and registering the implementation with the framework. The lifecycle of
an EventHandler can be chosen by the developer. Normally, an EventHandler lives as long as the simulation run.
It is notified before the beginning of each new iteration so that its state can be reset to listen to a new
iteration. This pattern can be used to collect information over all iterations. But if the purpose of
an EventHandler is to make a calculation based on one single iteration, it may be more natural
to create a new EventHandler instance for each iteration, query it for its result and discard it
after the iteration finishes. This can be done in a ControlerListener.
% =========================================================================================

\subsection{PlanStrategy}

Replanning in MATSim is specified by defining a set of weighted strategies. In each
iteration, each agent makes a draw from this set and executes the selected strategy. The
strategy specifies how the agent changes its behavior. Most generally, it is an operation on the plan memory
of an agent: It adds and/or removes plans, and it marks one of these plans as selected.

Strategies are implementations of the \verb$PlanStrategy$ interface. 
The most common case is that a \verb$PlanStrategy$ picks one plan from memory
at random, copies it, mutates it in some specific aspect, adds the mutated plan to the plan memory, and
marks this new plan as selected. MATSim provides a helper class which can be used
to implement this strategy template, where only the mutation operation has to specified.

The four most commonly used strategies shipped with MATSim are:

\begin{compactitem}
\item Select from the existing plans at random, which are weighted by their current score.
\item Mutate a random existing plan by re-routing all trips.
\item Mutate a random existing plan by randomly shifting all activity end times backwards or forwards and re-routing 
all trips.
\item Mutate a random existing plan by changing the mode of transport and re-routing one or more trips or tours.
\end{compactitem}

Routes are computed based on the traffic conditions of the previous iteration, which are measured
by means of an \verb$EventHandler$. Using the same pattern, your own \verb$PlanStrategy$ can use any data which
can be computed from the mobility simulation.

% =========================================================================================
\subsection{Scoring}
By default, MATSim uses the so-called Charypar-Nagel utility formulation to score plans. Its parameters
are configurable.
The code which maps a stream of mobility simulation events to a score for each agent is placed behind a factory
 interface and replaceable. However, replacing it means replacing the entire utility formulation. There is
 currently no mechanism for composing a utility formulation from contributions by different modules.
For instance, a module which simulates weather conditions would probably calculate penalties for pedestrians
  walking in heavy rain, and the Cadyts calibration scheme already uses utility offsets in its formulation. A modeller
  who wishes to compose a scoring function from the Charypar-Nagel utility, the rain penalty and the calibration offset
  needs to do this manually, in code, accessing the code of all three modules contributing to the score and (for instance) summing
  up their contributions. As of the writing of this chapter, this makes scoring in a way the least modular part of MATSim: It has to 
be re-defined, in code, for every combination of modules which contribute to the utility.

Keep in mind that scoring and replanning are not inherently coupled or automatically consistent with each other. 
Consider a scoring function which penalizes left-turns. This is straight-forward to program: You would 
iterate over every route an agent has taken. Looking at the \verb$Network$, you would calculate for each change of
links if you consider it a left-turn, and if so, add a (negative) penalty to the score. However, this would not by itself lead to a
solution where routes are distributed according to this scoring function. The reason is that the default replanning only proposes
 fastest routes, in other words, least-cost paths with respect to travel time. By default, the plan memory of an agent will only ever
 contain routes which have in one iteration been a fastest route. The behavior of the router is, in this case, inconsistent with the utility
 formulation.
   

% =========================================================================================
% ##################################################################################################################

% Local Variables:
% mode: latex
% mode: reftex
% mode: visual-line
% TeX-master: "../main"
% comment-padding: 1
% fill-column: 9999
% End: 
