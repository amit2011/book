% ##################################################################################################################
\section{Cottbus, Germany}
\label{ch:scenarios:cottbus}
\hfill \textbf{Author:} Joschka Bischoff
% ##################################################################################################################

The Cottbus scenario is used for traffic light simulation (see Chapter \ref{ch:signalslanes}). It is well described by \citet[][]{Grether_PhDThesis_2014}. The scenario data is generally available to the public.

The network is taken from openstreetmap data in summer 2010. It covers all streets within the city boundaries and main roads in the surrounding administrative district Spree-Nei�e. It is designed as a 100 \% sample. The population is based on the commuter statistic of the German federal employment agency for both Cottbus and Spree-Nei�e. As such, the population has only home-work-home plans that are spread over the usual commuting times, resulting in two peaks. Overall, 33'479 agents travelling exclusively by car are included. The scenario is overall not very busy, with the area not being known to have bigger congestion issues.

The scenario contains data for 22 traffic signals within the city centre which are based on the city's signal plans of 2009. The original junction layout is also depicted.

Public transit, though not part of the original scenario, is available based on the schedules of 2011. The population does, however, not currently use it. 

% ##################################################################################################################

% Andreas Neumann-Mail: frei verf�gbar. �V vorhanden, 100\% Szenario, das sich mit einigermassen geringem Aufwand rechnen l�sst. Grundlage f�r Dominik G.s Ampelpart