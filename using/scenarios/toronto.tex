% ##################################################################################################################
\section{Toronto}
\label{sec:toronto}
\hfill \textbf{Author:} Adam Weiss, Peter Kucireck, Khandker Nurul Habib

% ##################################################################################################################
\subsection{Study Area:}
The Greater Toronto and Hamilton Area (GTHA) is situated to the north west of Lake Ontario in the province of Ontario, forming Canada’s largest urban region. The GTHA’s current population is over 6.5 million, with projected growth to approximately 8.6 million by 2031. 

% =============================================================================================
\subsection{Population, Demand Generation and Activity locations}
The Transportation Tomorrow Survey (TTS) forms the basis of the travel demand to be used for the multimodal assignment simulation. The TTS is a retrospective telephone survey conducted in the GTHA every 5 years. The TTS samples just over 5\% of the GTHA households. The survey collects household socioeconomic and geographical characteristics, characteristics of each household member, and a full 24 h travel diary for each household member. The current MATSIM models use the TTS travel diary records to generate the plans file. There has also been some investigation into the integration of the Travel Activity Scheduler for Household Agents (TASHA) activity based model, which has been developed for the GTHA. Irrespective of the source of the demand data, both sources provide the traffic zone location of all activities. The Toronto implementation will then randomly distribute the activities around the traffic zone, resulting in unique x-y coordinates for each activity. Within the current implementation of MATSIM within Toronto, no development of MATSIM facilities has been attempted.

% =============================================================================================
\subsection{Network Development and Simulated Modes}  
The GTHA MATSIM implementation uses a preexisting planning level network for static user equilibrium assignment using the EMME traffic assignment software. This network is converted to a MATSIM network using a conversion tool, which can be found in the MATSIM Toronto playground. More recently, this network was merged with GTFS data for 5 of the 8 major regional transit agencies to allow for multimodal demand assignment.  

% =============================================================================================
\subsection{Calibration, Validation, Results}
The Toronto MATSIM implementation has been compared to the more conventional large-scale assignment models with varying degrees of success. While the work of \citet[][]{GaoWEtAl_TRR_2010}, found that travel time, travel distance, link flows and speeds were all reasonably comparable and in fact more plausible than those achieved through the EMME assignment. Conversely, work on transit assignment done first by \citet[][]{Kucirek_MastersThesis_2012} and then by \citet[][]{WeissEtAl_CJCE_2012} found that there were limitations associated with predicting line boardings based on different transit technologies and agencies, which utilized different fare structure, suggesting that further work to calibrate the multimodal assignment model is required. These issues are exasperated by the current implementations inability to distinguish between in vehicle dwell times and out of vehicle wait times, which ideally should be weighted differently, particularly given the climate and prominence of outdoor bus stops within the region. 

% =============================================================================================

% ##################################################################################################################