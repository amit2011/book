% ##################################################################################################################
\section{Toronto}
\label{sec:toronto}
\hfill \textbf{Author:} Adam Weiss, Peter Kucireck, Khandker Nurul Habib

% ##################################################################################################################
\subsection{Study Area}
The \gls{gtha} is situated to the north west of Lake Ontario in the province of Ontario, forming Canada’s largest urban region. The \gls{gtha}’s current population is over 6.5\,million, with projected growth to approximately 8.6\,million by 2031. 

% =============================================================================================
\subsection{Population, Demand Generation and Activity Locations}
The \gls{tts} forms the basis of the travel demand to be used for the \gls{multimodal} assignment simulation. The \gls{tts} is a retrospective telephone survey conducted in the \gls{gtha} every 5\,years. The \gls{tts} samples just over 5\,\% of the \gls{gtha} households. The survey collects household socioeconomic and geographical characteristics, characteristics of each household member, and a full 24\,hours travel diary for each household member. The current \gls{matsim} models use the \gls{tts} travel diary records to generate the plans file. There has also been some investigation into the integration of the \gls{tasha} activity based model, which has been developed for the gls{gtha}. Irrespective of the source of the demand data, both sources provide the traffic zone location of all activities. The Toronto implementation will then randomly distribute the activities around the traffic zone, resulting in unique x-y coordinates for each activity. Within the current implementation of \gls{matsim} within Toronto, no development of \gls{matsim} facilities has been attempted.

% =============================================================================================
\subsection{Network Development and Simulated Modes}  
The \gls{gtha} \gls{matsim} implementation uses a preexisting planning level network for static user equilibrium assignment using the \gls{emme} traffic assignment software. This network is converted to a \gls{matsim} network using a conversion tool, which can be found in the \gls{matsim} Toronto playground. More recently, this network was merged with \gls{gtfs} data for 5 of the 8\,major regional transit agencies to allow for \gls{multimodal} demand assignment.  

% =============================================================================================
\subsection{Calibration, Validation, Results}
The Toronto \gls{matsim} implementation has been compared to the more conventional large-scale assignment models with varying degrees of success. While the work of \citet[][]{GaoWEtAl_TRR_2010}, found that travel time, travel distance, link flows and speeds were all reasonably comparable and in fact more plausible than those achieved through the \gls{emme} assignment. Conversely, work on transit assignment done first by \citet[][]{Kucirek_MastersThesis_2012} and then by \citet[][]{WeissEtAl_CJCE_2012} found that there were limitations associated with predicting line boardings based on different transit technologies and agencies, which utilized different fare structure, suggesting that further work to calibrate the \gls{multimodal} assignment model is required. These issues are exasperated by the current implementations inability to distinguish between in vehicle dwell times and out of vehicle wait times, which ideally should be weighted differently, particularly given the climate and prominence of outdoor bus stops within the region. 

% =============================================================================================

% ##################################################################################################################