% ##################################################################################################################
\section{Toronto}
\label{sec:toronto}
\hfill \textbf{Authors:} Adam Weiss, Peter Kucireck, Khandker Nurul Habib

\editdone{This text has undergone the professional edit. Please no grammatical changes anymore! They are most-probably wrong.}

% ##################################################################################################################
\subsection{Study Area}
The \gls{gtha} is locate northwest of Lake Ontario, in the province of Ontario, forming Canada’s largest urban region. The \gls{gtha}’s current population is over 6.5\,million, with projected growth to approximately 8.6\,million by 2031. 

% =============================================================================================
\subsection{Population, Demand Generation and Activity Locations}
The \gls{tts} was the basis for travel demand used for the \gls{multimodal} assignment simulation. \gls{tts} was a retrospective telephone survey, conducted in the \gls{gtha} every 5\,years. The \gls{tts} sampled just over 5\,\% of \gls{gtha} households; the survey collected household socioeconomic and geographical data, characteristics of each household member and a full 24\,hours travel diary for each household member. Current \gls{matsim} models use the \gls{tts} travel diary records to generate the plans file. Integration of the \gls{tasha} activity based model, developed for the gls{gtha}, was also investigated. Irrespective of the demand data source, both sources provided the traffic zone location for all activities. The Toronto implementation then randomly distributed activities around the traffic zone, which resulted in unique x-y coordinates for each activity. Within the current \gls{matsim} implementation in Toronto, no \gls{matsim} facilities development has been attempted.

% =============================================================================================
\subsection{Network Development and Simulated Modes}  
The \gls{gtha} \gls{matsim} implementation used a pre-existing planning level network for static user equilibrium assignment,employing the \gls{emme} traffic assignment software. This network was converted to a \gls{matsim} network, using a conversion tool found in the \gls{matsim} Toronto playground. More recently, this network was merged with \gls{gtfs} data for 5 of the 8\,major regional transit agencies to allow for \gls{multimodal} demand assignment.  

% =============================================================================================
\subsection{Calibration, Validation, Results}
The Toronto \gls{matsim} implementation was compared to more conventional, large-scale assignment models with varying success. The work of \citet[][]{GaoWEtAl_TRR_2010} found that travel time, travel distance, link flows and speeds were reasonably comparable,  in fact more plausibe, than those achieved through the \gls{emme} assignment. Conversely, work on transit assignment, first done by \citet[][]{Kucirek_MastersThesis_2012} and then by \citet[][]{WeissEtAl_CJCE_2012}, found limitations associated with predicting line boarding figures; these were based on different transit technologies and agencies and utilized different fare structures, suggesting that further work to calibrate the \gls{multimodal} assignment model is required. These issues are exacerbated by the current implementation's inability to distinguish between in-vehicle dwell times and out-of-vehicle wait times; these should ideally be weighted differently, particularly given the climate and predominance of outdoor bus stops in the region. 

% ##################################################################################################################