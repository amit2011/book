% ##################################################################################################################
\section{Singapore}
\label{sec:singapore}
\hfill \textbf{Author:} Alexander Erath, Artem Chakirov

% ##################################################################################################################
The MATSim Singapore scenario \citet[][]{ErathEtAl_TechRep_FCL_forth} was implemented and is maintained at the Future Cities Laboratory, a research program of the Singapore-ETH Centre for Global Environmental Sustainability (SEC) and part of Singapore's National Research Foundation CREATE (Campus for Excellence and Technological Enterprise). The scenario covers the whole area of Singapore with a population of approximately 5 million inhabitants and includes traffic from and to neighboring Malaysia. Singapore provides an excellent study case for an agent- and activity based modeling approach: a fairly densely populated city with an extensive public transport infrastructure and advanced transportation and pricing policies. 

% ====================================================================================================
\subsection{Demand}
In the absence of a full-population census for Singapore, a synthetic population is generated based on data from the Household Interview Travel Survey (HITS) 2008 \citep[][]{Choi_JOUR_2010} and population breakdowns of Singapore’s population census 2010. The synthetic population was derived using the fitting and sampling method \citep{MuellerKAxhausen_TRB_2011}, where a reference sample of household and person records  is weighted using an iterative proportional fitting (IPF) technique, until the weighted sample matches marginal control totals from the census. In our case, the reference sample is the records form the travel survey, and the fitting technique is the entropy optimization method proposed by \citet[][]{BarGeraEtAl_TRB_2009}, implemented by Kirill Müller, IVT, ETH Zürich. Then the reference sample records are replicated through weighted sampling until the population total is met. 
 
Car ownership is modeled on a household level and driving licenses are assigned to individuals, using discrete choice methods. Given the high taxation of cars in Singapore, the model reflects that car ownership is much lower than in other developed nations. The model presented in \citet[][]{VanEggermondEtAl_IATBR_2012} includes not only socio-economic but also spatial variables and has proven to be essential to the MATSim Singapore model, leading to accurate mode choice and mode share predictions. 

Activity locations are defined on the level of individual buildings, with the information on building and facility types compiled from various sources such as the land-use master plan \citep[][]{URA_Rep_URA_2008}, government websites and online directories as well as points of interest information provided by NAVTEQ. In the absence of a business census, an innovative approach for identification of locations and corresponding number of work places has been developed. It draws from the full smart card data record of public transport journeys and enriches it with information on land-use and estimates of building floor space. In a first step, a probabilistic model is applied to a daily record of public transport journeys in order to identify types of activities performed between two subsequent public transport trips. Estimated and calibrated using HITS 2008 records, the model combines variables such as time of day, activity duration and land-use around each stop or station to ensure an accurate differentiation between home, work or other activities. After accounting for mode shares in 53\,different zones, an optimization technique employing accessibility computation is applied in order to distribute work activities to individual buildings. More details on the newly developed methodology and its practical application are reported in \citet[][]{ChakirovErath_IATBR_2012} and \citet[][]{OrdonezErath_TRR_2013}. 

The assignment of households to buildings is performed using detailed information on residential developments: for public housing which represent about 80\,\% of Singapore's residential building stock, information on the distribution of different dwell types is employed, while for privately owned condominiums, only information on the number of apartments per building is available. Work locations are assigned using a zone-based gravity model that uses the prior estimated number of work activities in each building as additional information for the distribution of workplaces within each zone. Activity chains are assigned based on their observed frequency in HITS, taking into account key socio-demographic parameters such as sex, age, occupation and income. Activity chains of type home~--~work~--~home are by far the most frequent, accounting for approximately 50\,\% of the trips.
Freight and cross border traffic as well as tourist travel demand are derived based on a set of origin destination matrices provided by the Singapore Land Transport Authority (LTA). These matrices are converted into special daily plans. Information on the temporal distribution of trips for freight is derived from loop detector data for freight and temporal attraction profiles of major tourist sites.

% ====================================================================================================
\subsection{Supply}
Using a semi-automatic map-matching algorithm~\ref{sec:networkeditor-singapore} a high-resolution navigation network provided by NAVTEQ is map-matched to and enhanced with lane and capacity information from the Land Transport Authority's (LTA) planning network. In absence of access to traffic signal cycle time data, traffic lights are not specifically modeled. Extensive attention was paid to the modeling of public transport due to the the importance of interaction between private and public transport in Singapore’s context of high density and limited space. Simulating dynamic effects such as bus bunching is crucial for obtaining realistic travel times and mode shares. Public transport network and schedule data provided by LTA includes bus and train routes,  as well as the location of stops and stations. This information has been matched to the road network using yet another map-matching algorithm presented by \citet[][]{Ordonez_HKSTS_2011, Ordonez_Webpage_2011_4}. More recently, the scenario got updated by using public transport schedule data as derived from public transport smart card data records \citet[][]{Fourie_TechRep_FCL_2014}. Such schedule information accounts for the actual vehicle dispatch frequencies and headways, which undergo continuous adjustments and in some cases can substantially deviate from the published schedule. Furthermore, additional features of the public transport simulation in Singapore’s model include advanced bus dwell time model \citep[][]{SunEtAl_TransResA_2014} as well as an approximation of the distance based public transport fare scheme.

Other modes, specifically walking and cycling are "teleported" with constant travel speeds and without any interaction with other users. 

% ====================================================================================================
\subsection{Behavioral parameters}
The behavioral parameters, specific to Singapore's context have been borrowed form the study by \citet[][]{LTA_unpub_2009} and used in conjunction with the widely applied Charypar-Nagel function for activity scoring \citet[][]{CharyparNagel_Transportation_2005}. Thereby same parameters have been used for all agents, neglecting heterogeneity of user preferences and values of time in the initial scenario implementation. Furthermore, no additional crowding penalties accounting for travelers' discomfort have been considered at this stage and the effects of public transport overcrowding are only taken into account with the physical vehicle capacity limitations as well as its implications on dwell time and occurrence of the bus bunching phenomenon. 

% ====================================================================================================
\subsection{Policy}
The MATSim model for Singapore also includes the Electronic Road Pricing (ERP) scheme featuring time and vehicle dependent road pricing. Based on two data sets indicating the location and time dependent price levels, prevailing tolls are specified for 73\,network links where toll gantries are installed as of February, 2012. To account for the widely available dedicated bus lanes, additional links that are attributed to be used exclusively by buses have been added to the network. The capacity of the respective existing links has been reduced accordingly, even if in some cases the exclusive use of dedicated lanes by buses is granted during the peak hours only. Such a simplified setup, insensitive to the time dynamic operation of the dedicated lanes, leads to an underestimation of actual road capacity during periods when bus lanes are also open to other motorized traffic. However, as most links featuring bus lanes consist of three or more lanes, the effect on modeled traffic conditions during off-peak hours appears to be low.

% ====================================================================================================
\subsection{Calibration and validation}
Road usage data is available for around 200\, 
%\textbf{EXACT 200?} 
count stations in hourly intervals. The availability of public transport smart card data provides an additional dimension for validation. In particular opening of new mass rapid transit (MRT) lines since setting up the model in 2012 presents a unique opportunity for comparison the observed ridership with predicted ridership in the model. However, systematic calibration and detailed validation are yet to be conducted.

% ##################################################################################################################