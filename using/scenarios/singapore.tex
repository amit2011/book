% ##################################################################################################################
\section{Singapore}
\label{ch:scenarios:singapore}
\hfill \textbf{Author:} Alexander Erath?

The Singapore scenario is build at the Future Cities Laboratory in Singapore embedded in the Singapore National Research Foundation initiative CREATE (Campus for Excellence and Technological Enterprise). The scenario is detailed by \citet[][]{ErathEtAl_TechRep_FCL_forth, Erath_unpub_UniSeoul_2011}.

% --------
\paragraph{Associated projects:} 
The scenario is built for the MATSim Singapore project presented in Section \ref{sec:singaporeproject}.

% --------
\paragraph{Study area:} 
The scenario covers the whole republic of Singapore with its approximately 5 million inhabitants.

% --------
\paragraph{Population and demand generation:} 
For population generation an IPF-approach is adopted. A full-population census is not available for Singapore. Demand is derived from a national travel diary survey (Household Interview Travel Survey) reported by \citet[][]{Choi_JOUR_2010} and containing about 11'000 households in Singapore. Home and work locations are assigned by employing a gravity-model-like approach. Freight trips and non-permanent resident inhabitants' trips are generated from origin destination matrices provided by the Singapore Land Transport Authority (LTA).

% --------
\paragraph{Activity Locations:} 
Activity locations are defined at a single building level. Various sources as described in Section 4.1 of \citet[][]{ErathEtAl_TechRep_FCL_forth} have been merged. Workplace capacities are estimated by \citet[][]{OrdonezErath_TRR_2013}. This is required for assigning the fixed activity locations to the agents.

% --------
\paragraph{Network:} 
For the Singapore model both a planning network (provided by LTA) and a Navteq navigation network are readily available. These two networks are combined for the Singapore scenario by a semi-automatic map-matching algorithm. Public transport routes are matched to the network by another map-matching algorithm presented by \citet[][]{Ordonez_HKSTS_2011, Ordonez_Webpage_2011_4}. This enables interaction between public transport and individual traffic.

% --------
\paragraph{Modes:} 
The scenario simulates car and public transport. The modes walk and bike are ``teleported''. Public transport schedules were derived from General Transit Feed Specification (GTFS) created by Google.

% --------
\paragraph{Calibration and validation:}
For validation road count data on an hourly base are available for 200 stations in Singapore.

% --------
\paragraph{Simulation quality and achieved results:}

% --------
\paragraph{Miscellaneous, important to mention:}
Traffic lights are not yet included in the model due to missing data on signal schedules.
% ##################################################################################################################