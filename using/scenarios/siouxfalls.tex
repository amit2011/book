% ##################################################################################################################
\section{Sioux Falls}
\label{ch:scenarios:siouxfalls}
\hfill \textbf{Author:} Artem Chakirov

% ##################################################################################################################
The Sioux Falls scenario aims to provide a convenient test-case by combining a fully dynamic demand fitted with realistic socio-economic and demographic attributes with a small scale road network featuring an integrated public transportation system. Based on the Sioux Falls road network commonly used for tests and demonstration purposes in the transportation literature \citep[][]{BarGera_TNTP_Webpage_2013}, it attempts to allow quick and convenient experiments of new policy or software implementations with \gls{matsim} on a heterogeneous agent population with a high degree of spatial resolution, but at the same time without significant computational requirements. However, it is important to stress that despite the use of real world data for the generation of the enriched Sioux Falls scenario, it does not aim to replicate the real City of Sioux Falls in South Dakota, US and remains a fictitious test-bed. Detailed report on scenario generation and its characteristics is provided by \citet[][]{ChakirovFourie_TechRep_FCL_2014} and can also be found at \url{www.matsim.org/scenario/sioux-falls}. 

% ====================================================================================================
\subsection{Demand}
A realistic, socio-economically and demographically diverse demand population with  heterogeneous use preferences is crucial for unlocking the full potential of an agent-based simulation such as \gls{matsim}. However, the generation of a disaggregated demand description on individual and household levels that closely resembles reality is challenging; not only in terms of trip origins and destinations, but also with respect to the relation of travel patterns and socio-demographic characteristics of travelers.

In order to address this challenge for the case of the Sioux Falls scenario, and represent the household structure, demographic profile and income distribution as realistically as possible, a synthetic population of households using the entropy optimization technique of \citet[][]{BarGeraEtAl_TRB_2009} was generated. It matches the aggregate distribution of demographic attributes (age, sex and household income) recorded during the 2010 US Census for the 27\,census tracts inside and adjoining the city center of the City of Sioux Falls and is composed of household and person records taken from the (anonymous) 5-year sample (2007-2011) of the American Community Survey, covering 5\,\% of all households.

In order to keep the scenario accessible and to facilitate interpretation and understanding of the possible effects of policies under study, only two simple activity chains are initially included: \"home – work – home\" and \"home – other – home\". Activity locations are identified using the data set of the building stock provided by the City of Sioux Falls \gls{gis} division. The home location of each household is assigned randomly to a residential unit within the tract the household belongs to. As no information on the real number and distribution of work places within the relevant area is easily accessible, the static \gls{od} matrix from \citet[][]{LeBlancEtAl_TransRes_1975} are taken as an indicator of workplace attraction for each zone. Subsequently, the assignment of work places to individual workers as well as locations of secondary (other) activities is performed using a parameter free radiation model presented by \citet[][]{SiminiEtAl_NAT_2012}.

In order to exploit the full potential of a disaggregated demand and to add another degree of realism to the scenario, car ownership on the household level is modeled using an ordered probit model, presented by \citet[][]{GiulianoDargay_TransResA_2006} and based on the \gls{npts}~1995. In addition to socio-demographic characteristics of a household (number of adults, children, pensioners, household income), the model uses attributes of residential location (population density, public transport access and dwelling type), which allows better accounting for specific characteristics of the Sioux Falls scenario and its area-wide bus network. 

% ====================================================================================================
\subsection{Supply} 
A realistic transportation test network should ensure sufficient complexity of travelers’ choice dimensions while limiting  computational effort. To this end, the Sioux Falls test network was introduced by \citet[][]{MorlokEtAl_ResRep_org-fhwa_1973} and later adapted as a benchmark and test scenario in many publications (see \citet[][]{ChakirovFourie_TechRep_FCL_2014} for overview). The structure of this network captures the major arterial roads of the City of Sioux Falls in South Dakota, but was never intended to replicate the real city or all characteristics of its transportation system, such as travel times or modal split. The original network is comprised of 76\,arcs, 24\,nodes and 552\,\gls{od} pairs. For this scenario the capacities of the roads are adjusted according to values provided by the Highway Capacity Manual \citet[][]{HCM_2010} and other related research publications \citep[e.g.,][]{NgCFSmall_Transportation_2012}. The public transportation network added to the scenario includes 5\,bus lines, as initially proposed by \citet[][]{AbdulaalLeBlanc_TransScience_1979}, with bus stops placed at a constant distance of 600\,meters away from each other. 

Due to the design of \gls{matsim}’s queue simulation, agents are only handled at the beginning and end of each network link, and cannot enter or leave a link along its length. Therefore, origins and destinations located along very long links will lead to a loss of spatial detail, as all origins and destinations along the length of the link are effectively assigned the same coordinate. Consequently, to improve the level of spatial detail, all links of the Sioux Falls network are evenly split into smaller links with maximal length of 500\,meters each. Following this operation, the number of nodes was increased to 282 and number of links to 334, without however changing the effective network topology.

In addition to car and bus modes, walking as "teleported" mode with constant travel speeds and without any interaction with other users is used as the non-motorized mode of transportation. 

% ====================================================================================================
\subsection{Behavioral Parameters}
The behavioral parameters used in the utility functions are based on the estimated demand model for Sydney by \citet[][]{TirachiniHensherRose_TransResB_2014}. Before applying the parameters in an activity-based context, the time related parameters need to be adjusted to account for utility gained from activity performance. Thereby, in order to provide a value for marginal utility of performing an activity, the travel mode with smallest the disutility is set as a baseline, under the assumption that traveling with this mode is as good or as bad as idling and doing nothing. The corresponding parameters are split into opportunity costs of time and a mode-specific disutility of traveling, as has been done in previous \gls{matsim} related publications as \citet[e.g,][]{KickhoeferEtAl_Transportation_2011}. 

% ====================================================================================================
\subsection{Results, Drawbacks and Outlook:}
The stability and performance of the Sioux Falls scenario has been tested using two sets of activity timing constrains as well as 5\,different random seeds, which all have delivered stable and realistic results. Furthermore, \citet[][]{ChakirovFourie_TechRep_FCL_2014} also investigate the existence and the hysteresis characteristic of the \gls{mfd}, as discussed in \citet[][]{GeroliminisDaganzo_TRB_2007, GeroliminisDaganzo_TransResB_2008, GeroliminisSun_TransResA_2011}. 

However, recent experience has shown certain drawbacks of the coarse network representing only major arterial roads and neglecting minor neighborhood and collector road links. With an elaborate synthetic population and high demand peaks during rush hours, the network appears to be sensitive to network breakdowns under high loading conditions. 

Along with the coarse road network, the coarse level of public transport network and therewith its low level of accessibility for parts of the population represents another drawback, in particular relevant to simulation and evaluation of policies sensitive to or requiring a certain share of public transport users. 

Substituting, the original Sioux-Falls networks with a finer network obtained from the crowd-sourced \gls{osm}, and potentially adding additional public transport lines to it, would allow to address the aforementioned weakness of this scenario. However, a different set of drawbacks arises from it and requires further attention. First, the significantly larger number of links and nodes in the network increases the time and resources for routing and dynamic queue simulation and threatens to erase the advantages of a small scale network. The extended simulation times can be tackled with the new pseudo-simulation methodology, currently developed by \citet[][]{FourieEtAl_TRR_2013}.
Second, the increase in total network capacity leads to the reduction or even disappearance of congestion during peak hours. Thereby, inclusion of freight and through traffic into the scenario can increase the degree of realism and cater for congested conditions during the peak-hours. 

% ##################################################################################################################
