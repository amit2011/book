% ##################################################################################################################
\section{Nelson Mandela Bay}
\label{sec:nelsonMandelaBay}
\hfill \textbf{Author:} Johan W. Joubert

% ##################################################################################################################
Nelson Mandela Bay metropole is in the Eastern Cape province of South Africa and includes the cities of Port Elizabeth and Uitenhage, with a population of approximately 1.2 million inhabitants.

The reasoning behind the development of a scenario for this area was a complexity one. We needed an area where we could experiment with various modules and elements that MATSim offered, yet where the are is less complex than the megacity region of Gauteng. The Nelson Mandela Bay case was attractive as it still has a substantially large population; had only one formal bus operator (Algoa Bus Company) and one passenger rail operator (Metrorail); and it displayed the characteristic apartheid urban form of many South African cities and towns where many low-income commuters resided on the outskirts of the spatially sprawled cities.

The population was at first generated from the 2001 census, and later revised and updated to the 2011 census data. Travel demand was inferred from the 2006 travel diary conducted in the metro. The process of generating the synthetic population is described in detail on the MATSim \href{https://matsim.atlassian.net/wiki/display/MATPUB/South+Africa}{Confluence} site. The population is generated in the form of entire households, using the Multi-Level Iterative Proportional Fitting (MLIPF) as published by \citet[][]{MuellerKAxhausen_LATSIS_2012}. The households are also assigned to buildings based on the census description.

It was the first South African scenario that included private cars, freight and detailed public transport. The inclusion of the prominent minibus taxis, a form of paratransit in South Africa and many developing countries, was done in the Nelson Mandela Bay area and is reported in \citet[][]{Roeder2013MasterMinibus} and \citet[][]{NeumannEtAl2014MinibusRSA}.

% ##################################################################################################################