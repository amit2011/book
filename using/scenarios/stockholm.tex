% ##################################################################################################################
\section{Stockholm}
\label{sec:stockholm}
\hfill \textbf{Author:} Joschka Bischoff

% ##################################################################################################################
The Stockholm scenario was created as a student project at TU Berlin in Summer 2014. As several groups worked on the project, the common base are a synthetic population from census data, an OSM based network and counts data.

The network is taken from OSM data in 2013. Within the city, all roads are used, whereas in outlying regions only mayor roads are part of the network. The demand consists of home-work-home-plans only. The population sample size is, depending on the student group, between 1 and 5\,\%. Agents are using car and (pseudo) public transit.

Count data for the morning peak along a mayor road, the E4, was used to calibrate the scenario. This calibration was handled differently by the groups; some just added traffic, others tried to imitate the Stockholm toll. Further documentation about the scenario is available in German language. 

% ##################################################################################################################
