% ##################################################################################################################
\section{Stockholm}
\label{sec:stockholm}
\hfill \textbf{Author:} Joschka Bischoff

\editdone{This text has undergone the professional edit. Please no grammatical changes anymore! They are most-probably wrong.}

% ##################################################################################################################
The Stockholm scenario was created as a student project at TU Berlin in summer, 2014. Because several groups worked on the project, the common base was a census data synthetic population, an \gls{osm}-based network and counts data.

The network was taken from \gls{osm} 2013 data. Within the city, all roads were used; in outlying regions, only mayor roads were included in the network. Demand consisted of home-work-home-plans only. The population sample size was---depending on the student group---between 1 and 5\,\%. Agents used car and (pseudo) public transit.

Count data for the morning peak along a mayor road, the E4, was used to calibrate the scenario. This calibration was handled differently by the groups; some just added traffic, others tried to imitate the Stockholm toll. Further scenario documentation is available in German. 

% ##################################################################################################################
