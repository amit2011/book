% ##################################################################################################################
\section{Brussels}
\label{sec:brussels}
\hfill \textbf{Author:} Daniel Röder

\editdone{This text has undergone the professional edit. Please no grammatical changes anymore! They are most-probably wrong.}

% ##################################################################################################################
The \gls{matsim} scenario for Brussels was developed as part of the \gls{sustaincity} project. This project's goal was to couple an urban land use model, in this case \gls{urbansim}, with the \gls{matsim} mobility simulation, to evaluate transport policy impact on urban land use and vice versa. A detailed description of this coupling is given by \citet{Nicolai2013PhD} and others. A detailed description of the scenario development is found in \citet{RoederNagel2013SketchPlanningBrussels}.

The scenario covered the greater Brussels area in Belgium; input data was derived from two main sources. The population was directly generated from the \gls{urbansim} model, covering a total of 860\,214 persons. At home- and at work-locations (per person) were given and converted into a daily home-work-home plan. For computational reasons, a randomly-drawn population sample of one percent was used. \gls{osm} was sourced for the street network generation, which consisted of 10\,861 nodes and 19\,830 links, i.e.,\ using mainly the trunk road network.

For the modeling of public transport, two different approaches were tested:  first, the \gls{matsim} default approach for scenarios where no detailed transit schedule is available, based on either: beeline distance and average speed, or network-based freespeed travel times and a designated factor. The second approach was not part of the \gls{matsim} core during the project, but was available as a contribution (\lstinline|matrixBasedPtRouter|). It was based on \gls{od} travel time matrices between transit stops, i.e.,\ travel times for all relations were computed in a pre-process. The travel times can based on a real-world-schedule or certain assumptions which can take spatial coverage into account. Advantages of this model are obvious; on one hand, it may depict spatial coverage with public transport supply---here, distance to the next transit stop influences travel time. On the other hand, it may depict the real network, i.e.,\,routes and lines and possible waiting times for switching. Both approaches were compared against travel times and mode share measures from a \gls{saturn} model. Since the matrix-based approach came closer to this model, further investigations were based on that.

To evaluate the model's sensitivity to certain policies, a cordon toll scenario was set up, where a toll is charged between 6 and 10\,am every time a car passed a cordon border. i.e.,\ every time a car entered a link crossing a cordon border defined by the Brussels freeway ring. Accessibility was calculated and compared for both scenarios.  \citet{RoederNagel2013SketchPlanningBrussels} provides a detailed analysis.

% ##################################################################################################################
