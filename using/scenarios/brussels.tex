% ##################################################################################################################
\section{Brussels}
\label{sec:brussels}
\hfill \textbf{Author:} Daniel Röder

% ##################################################################################################################
The \gls{matsim} scenario for Brussels was developed in the project \gls{sustaincity}. The overall goal of this project was to couple an urban landuse model, in this case \gls{urbansim}, with the mobility simulation \gls{matsim}. This was done to evaluate the impact of transport policies to urban landuse and vice versa. A detailed description of the coupling of both models is e.g.,\,given by \citet{Nicolai2013PhD}. A detailed description of the development of the scenario is given by \citet{RoederNagel2013SketchPlanningBrussels}.

The scenario covers the greater Brussels area in Belgium. The input data for the scenario is derived from two main sources. The population is directly generated from the \gls{urbansim} model, which covers a total number of 860\,214 persons. Per person a home- and a work-location is given and converted into a daily home-work-home plan. For computational reasons a randomly drawn population sample of one percent is used. \gls{osm} is used for the generation of the street network, which consists of 10\,861 nodes and 19\,830 links, i.e.,\,mainly the trunk road network was used.

For the modeling of public transport two different approaches are tested. The first approach is the \gls{matsim} default approach for scenarios where no detailed transit schedule is available. It is based on either beeline distance and average speed or network-based freespeed travel times and a certain factor. The second approach was not part of the \gls{matsim} core during the project but available as a contribution (\lstinline|matrixBasedPtRouter|). It is based on \gls{od} travel time matrices between transit stops, i.e.,\,travel times for all relations are computed in a pre-process. The travel times may be based on a real-world-schedule or certain assumptions which may take spatial coverage into account. The advantages of this model are obvious. On the one hand it may depict the spatial coverage with public transport supply and thus the distance to the next transit stop influences the travel time. On the other hand it may depict the real network, i.e.,\,routes and lines and possible waiting times for switching. Both approaches were compared against travel times and mode share measures from a \gls{saturn} model. Since the matrix-based approach comes closer to this model further investigations are based on this 
approach.

To evaluate the sensitivity of the model to certain policies a cordon toll scenario was set up. The toll is charged between 6 and 10\,am every time a car passes the border of the cordon, i.e.,\,every time a car enters a link which crosses the border of the cordon which is defined by the Brussels freeway ring. For both scenarios accessibility was calculated and compared against each other. A detailed analysis is given by \citet{RoederNagel2013SketchPlanningBrussels}.

% ##################################################################################################################
