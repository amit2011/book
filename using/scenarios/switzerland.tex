% ##################################################################################################################
\chapter{Switzerland}
\label{ch:switzerland-scenario}
\hfill \textbf{Author:} Andreas Horni

\editdone{This text has undergone the professional edit. Please no grammatical changes anymore! They are most-probably wrong.}

% ##################################################################################################################
The Switzerland scenario was initially created for the project Westumfahrung \citep[][]{BalmerEtAl_ResRep_bdktzrh_2009} and serves as the base for the very frequently used Zürich scenario (Chapter~\ref{sec:zhscenario}). 

Two main branches can be distinguished. The first, older one is based on a one-to-one translation of the Swiss population census \citep[][]{BfS_VZ_2000}; the second applies approaches from the \gls{ipf} family, reported by \citet[][]{MuellerKAxhausen_TechRep_IVT_2013, Mueller_unpub_LATSIS_2012, Mueller_unpub_ETC_2011, Mueller_unpub_STRC_2011, Mueller_unpub_IATBR_2012} to generate population.

The scenario's study area covered all of Switzerland. Due to administrative borders, no demand and supply data were available for adjoining countries, which, then and now, leads to boundary effects; studies focusing on Swiss border areas are difficult.

The population was derived from the Swiss Census of Population~2000 \citep[][]{BfS_VZ_2000}. The complete Swiss population was modeled, resulting in around 7.5\,million agents. 

This population's home locations were given at hectare level and work locations were specified at municipality level from commuter matrices, a component of the Swiss Census of Population~2000 \citep[][p.35]{BalmerEtAl_ResRep_bdktzrh_2009}. A very good overview, in German, of the population generation, its initial individual demand and activity locations can be found in \citet{MeisterEtAl_SVT_2009}. Further information is given in \citet[][]{CiariEtAl_STRC_2008, MeisterEtAl_WCTRS_2010, BalmerEtAl_ResRep_bdktzrh_2009, BalmerEtAl_ResRep_datapuls_2010, BalmerEtAl_HEUREKA_2008}.

Travel demand was basically taken from the 2000 and 2005 National Travel Surveys \citep[][]{BfS-MZ2005_manual_2006} (Swiss microcensus), although this sample substantially underestimated freight traffic and ignored cross-border traffic of non-Swiss residents. Freight traffic for Switzerland was missing at this time (except Zürich, see next chapter). Cross-border traffic was derived from mode-specific, hourly origin-destination matrices given by \citet[][]{VrticEtAl_ResRep_UVEK_2007}. These were disaggregated to around 600\,000 individual \gls{matsim} plans for the whole country, which contain the cross-border traffic originating \emph{outside} Switzerland. Non-Swiss, cross-border traffic starting in Switzerland was supposed to be negligible. 

The activity location data set, comprising home, work, education, shopping and leisure locations, was also derived from the 2000~Swiss Census of Population and the 2001~Federal Enterprise Census \citep[][]{SwissEnterpriseCensus_manual_2001}, providing hectare level information. Facility generation was described in \citet[][p.33]{BalmerEtAl_ResRep_bdktzrh_2009}.

For car traffic, navigation networks from Teleatlas \citep[][]{MultiNet_Webpage_2010} and \gls{navteq} \citep[][]{Navteq_2011} were available. The most-used network was the planning network derived from from the Swiss National Transport Model \citep[][]{VrticEtAl_BiegerEtAl_2003}.

The public transport simulation network was derived from the National Transport Model of the \gls{uvek}, described by \citet[][]{VrticFroehlich_ResRep_UVEK_2010}. 

The scenario simulated car and public transport; schedules for public transport were given at the municipality level. Fine-granular schedules were not available then, but were in preparation. The modes walk and bike were usually ``teleported''. 

Calibration was mainly performed for modal split and distance distributions; utility function values were set accordingly.

For validation, count data on city level, cantonal level and national level \citep[][]{ASTRA_Webpage_2006} were available from various sources, resulting in 600\,links measured for Switzerland. An average working day (Monday to Thursday, excluding public holidays) was used for comparisons in current projects.

% ##################################################################################################################