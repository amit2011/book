% ##################################################################################################################
\section{Switzerland}
\hfill \textbf{Author:} Andreas Horni

% ##################################################################################################################
The Switzerland scenario was initially created for the project Westumfahrung \citep[][]{BalmerEtAl_ResRep_bdktzrh_2009}. It serves as the base for the very frequently used Zürich scenario (Section \ref{sec:zhscenario}). 

Two main branches can be distinguished. The first and older one is based on a one to one translation of the Swiss population census \citep[][]{BfS_VZ_2000}, whereas the second one applies approaches from the family of \gls{ipf} reported by \citet[][]{MuellerKAxhausen_TechRep_IVT_2013, Mueller_unpub_LATSIS_2012, Mueller_unpub_ETC_2011, Mueller_unpub_STRC_2011, Mueller_unpub_IATBR_2012} to generate the population.

The scenario's study area covers all of Switzerland. Due to the administrative borders no data for demand and supply are yet available for the adjoining countries, which can lead to boundary effects. This means that studies focusing on the Swiss border area are difficult.

The population is derived from the Swiss Census of Population~2000 \citep[][]{BfS_VZ_2000}. The complete Swiss population is modeled which results in around 7.5\,million agents. 

This population's home locations are given at hectare level and work locations are known at municipality level from the commuter matrices, a component of the Swiss Census of Population~2000 \citep[][p.35]{BalmerEtAl_ResRep_bdktzrh_2009}. A very good overview in German of the population generation, its initial individual demand and activity locations can be found in \citet{MeisterEtAl_SVT_2009}. Further information is given in \citet[][]{CiariEtAl_STRC_2008, MeisterEtAl_WCTRS_2010, BalmerEtAl_ResRep_bdktzrh_2009, BalmerEtAl_ResRep_datapuls_2010, BalmerEtAl_HEUREKA_2008}.

Travel demand is basically taken from the National Travel Survey for the years 2000 and 2005 \citep[][]{BfS-MZ2005_manual_2006} (Swiss microcensus). This sample however, substantially underestimates freight traffic and ignores cross-border traffic of non-Swiss residents. Freight traffic for whole of Switzerland is to date missing (but not for Zürich, see next section). Cross-border traffic is derived from mode-specific hourly origin-destination matrices given by \citet[][]{VrticEtAl_ResRep_UVEK_2007}, which are disaggregated to around 600\,000 individual \gls{matsim} plans for all of Switzerland, which contain the cross-border traffic that originates \emph{outside} Switzerland. Non-Swiss cross-border traffic starting in Switzerland is supposed to be negligible. 

The activity location data set, comprising home, work, education, shopping and leisure locations, is also derived from the Swiss Census of Population 2000 and the Federal Enterprise Census 2001 \citep[][]{SwissEnterpriseCensus_manual_2001} providing hectare level information. The generation of facilities is described in \citet[][p.33]{BalmerEtAl_ResRep_bdktzrh_2009}.

For car traffic, navigation networks from Teleatlas \citep[][]{MultiNet_Webpage_2010} and \gls{navteq} \citep[][]{Navteq_2011} are available. The most often used network is the planning network derived from from the Swiss National Transport Model \citep[][]{VrticEtAl_BiegerEtAl_2003}.

The network for public transport simulation is derived from the National Transport Model of the \gls{uvek} described by \citet[][]{VrticFroehlich_ResRep_UVEK_2010}. 

The scenario simulates car and public transport. The schedules for public transport are given at the municipality level. Fine-granular schedules are not yet available, but are in preparation. The modes walk and bike are usually "teleported". 

Calibration is mainly done for modal split and distance distributions. Utility function values are set accordingly.

For validation, count data on city level, cantonal level and national level \citep[][]{ASTRA_Webpage_2006} are available from various sources resulting in 600\,links measured for Switzerland. An average working day (Monday to Thursday, excluding public holidays) is used for comparisons in current projects.

% ##################################################################################################################