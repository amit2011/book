% ##################################################################################################################
\section{Barcelona}
\label{sec:barcelona}
\hfill \textbf{Author:} Miguel Picornell

% ##################################################################################################################
The Barcelona scenario is one of the three case studies (together with London and Zurich) carried out under the framework of the EUNOIA project (\url{www.eunoia-project.eu}). The main goal of the Barcelona case study is to evaluate the impact of different public bike-sharing schemes in the city. The study area covers the metropolitan area of Barcelona, with special focus on the city center, where public bike-sharing stations are located. For this study a novel bike-sharing module was developed by ETH Zurich.

\subsection{Transport Supply: Network and Public Transport}
The road network has been extracted from the TRANSCAD model used by the city of Barcelona.  Public transport supply has been also considered, comprising: bus, underground, tram, train and bike-sharing. Information about stops and schedules has been obtained from the public information available at the Barcelona Open Data platform as well as from the Barcelona transport authority website. 

\subsection{Transport Demand: Population} 
Agent plans have been defined using anonymised mobile phone registers (CDRs). From mobile phone data it is possible to identify places where the agents perform activities and their corresponding trips derived from those activities. Activities have been classified as ‘home’, ‘work’ and ‘other’ (including as ‘other’ leisure, shopping, etc.). A sample of around 15\% of the population has been used in the simulation. Walk, cycling, public transport and car are the modes included in the simulation model.

\subsection{Calibration and Validation}
Different data sources have been used to calibrate and validate the model. First, in order to validate the agent plans obtained from mobile phone data, results were compared to the EMEF survey (Barcelona’s annually transport survey), leading to conclude that mobile phone data provide similar information than traditional surveys. Additionally, agent’s utility function has been calibrated using the modal split from the EMEF survey and road counts.

\subsection{Results and More Information}
At the time this summary was written, the calibration process ongoing. More detailed information about the scenario and main results can be found at:
\url{www.eunoia-project.eu/publications/} (project deliverables/Report on Case Study 3: Barcelona).

% ##################################################################################################################
