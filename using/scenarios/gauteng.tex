% ##################################################################################################################
\section{Gauteng}
\label{sec:gauteng}
\hfill \textbf{Author:} Johan W. Joubert

% ##################################################################################################################
Gauteng is a landlocked province in South Africa, with  three main metropolitan areas: the City of Johannesburg, City of Tshwane (formerly Pretoria), and Eurhuleni. Although the province account for less than 3\% of the country's surface, it is the country's economic hub and contributes a third of the country's Gross Domestic Product (GDP). The 2011 census reported a population of 12.2 million inhabitants, a quarter of the South African population. 

The first Gauteng scenario was developed in 2008/9 and appeared in \citet[][]{Fourie2009MastersThesis} and \citet[][]{FourieJoubert_SATC_2009}.  The populations was synthesized from the census 2001 data, and travel demand was inferred from the 2003 National Household Travel Survey (NHTS). At first the network was created from a proprietory source made available for research purposes, but has subsequently been replaced with a much richer \emph{OpenStreetMap} network.

Early comparisons already showed that the Gauteng MATSim scenario provided much richer results than the state of practice, four-step models that were available at the time~\citep[][]{Fourie_SATC_2010}. The scenario was also extended to include freight vehicles~\citep[][]{JoubertJEtAl_TRR_2010}.

With the introduction of an open-road tolling scheme referred to as the \emph{Gauteng Freeway Improvment Project} (GFIP), the scenario was used to study the diversion patterns of different road user groups. The population was extended to included background traffic in the form of public transport (buses and minibus taxis) and external through-traffic. This data was taken from Saturn OD-matrices made available by the sponsor, the South African National Roads Agency Limited (SANRAL). The impact of the tolling scheme using vehicle-specific values of time, and a complex toll pricing regime was reported in \citet[][]{NagelKickhoeferJoubert2014HeterogeneousVoTsPROCEDIA}.

The most recent update to the synthetic population generation for the Gauteng scenario is documented on MATSim's \href{https://matsim.atlassian.net/wiki/display/MATPUB/South+Africa}{Confluence} site.

% ##################################################################################################################