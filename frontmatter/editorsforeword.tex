\chapter*{Editors' Foreword}
% ##################################################################################################################
Developing complex software for over a decade with a heterogeneous group of engineers and scientists with widely different skill levels and expertise across multiple locations spread around the world requires dedication and mechanisms which are unusual for a university environment. 

This book is one of these mechanisms. It allows us collectively to take stock and to present the state-of-the-system in a coherent fashion to ourselves and to all those, who might be interested in the approach. It highlights the basics for the student, who wants to undertake a small  first research project as part of her degree, provides description of the main functionalities in detail for the engineer setting up MATSim for a policy analysis and finally places the approach into the overall theoretical background of complex systems as they have been developed in computer science and physics. 

The choice of the e-book format was a positive choice, as it allows us to keep the book up-to-date with future chapters, revisions and, if necessary, errata. Equally importantly it allows you, the readers, to select those parts, which are relevant to your needs. 

The book comes at an important time for the system. While for most of the first decade its use had been limited to the original developers and users in Berlin and Zurich, it is now much more widely adopted around the world, as we document in the chapter where we collect the contributions on the scenarios known to us. 

This term will occur again and again and in MATSim context means the combination of a specific agent population, their initial plans and their locations (home, work, education), the network and facilities on which they compete in time-space for their slots, and the modules, i.e.\, behavioral dimensions, which they can adjust during their search for equilibrium. Within this scenarios the user can experiment and explore with the parameters of the behavioral utility function, the sampling rate of the population between 1\,\% and 100\,\%, and the parameters of the algorithms, for example of the share of the sample engaged in replanning in any iteration and behavioral dimension or the exact settings involved to avoid gridlock due to artifacts of the traffic flow simulation. The creation of a \gls{scenario} is a substantial effort and the \gls{framework} makes a number of tools available to accelerate it: population synthesizers, network editors, network converters between popular formats and the MATSim representation (e.g.,\,OSM, GTF), semi-automatic network matching to join information etc .

The number of colleagues has been large, and many of them are contributors to this book, but as not all them are, here a list of those past and present involved in Berlin, Singapore and Zürich:  

\begin{multicols}{3}
Amit Agarwal \\
Milos Balac  \\
Dr. Michael Balmer\mbox{*} \\
Henrik Becker \\
Joschka Bischoff \\
Patrick Bösch \\
Dr. David Charypar \\
Dr. Nurhan Cetin  \\
Artem Chakirov \\
Dr. Yu Chen \\
Dr. Francesco Ciari \\
Dr. Michal Maciejewski \\
Dr. Christoph Dobler \\
Thibaut Dubernet\mbox{*} \\
Dr. Alexander Erath \\
Dr. Matthias Feil \\
Prof. Gunnar Flötteröd \\
Pieter J. Fourie\mbox{*} \\
Dr. Christian Gloor \\
Dr. Dominik Grether \\
Dr. Jeremy K. Hackney \\
Dr. Johannes Illenberger \\
Dr. Johan W. Joubert\mbox{*} \\
Ihab Kaddoura \\
Dr. Benjamin Kickhöfer \\
Dr. Gregor Lämmel\mbox{*} \\
Nicolas Lefebvre \\
Dr. Fabrice Marchal \\
Alejandro Marmolejo \\
Dr. Konrad Meister \\
Dr. Manuel Moyo Oliveros \\
Kirill Müller \\
Dr. Andreas Neumann \\
Dr. Thomas Nicolai \\
Dr. Benjamin Kickhöfer \\
Sergio A. Ordóñez Medina \\
Dr. Bryan Raney \\
Dr. Marcel Rieser\mbox{*} \\
Dr. Nadine Rieser-Schüssler \\
Dr. Daniel Röder \\
Mohit Shah \\
Lijun Sun \\
Alexander Stahel \\
Dr. David Strippgen \\
Dr. Basil Vitins \\
Michael Van Eggermond \\
Dr. Rashid Waraich\mbox{*} \\
Dominik Ziemke \\
Michael Zilske\mbox{*} \\
\end{multicols}
% 
We hope, that we can acknowledge the contributions of more colleagues from other groups, in future versions of this book and the software, beyond what we are lucky to do in this edition.   

Special thanks is due to the members of the "committee" (marked with an \mbox{*}), which is taking the final decisions on the shape of the MATSim core, the allocation between core and contributions and the software engineering of the framework.

This committee generally meeting virtually is a further mechanism of the coordination, but without regular face-to-face meetings one cannot maintain the trust and understanding necessary. An annual user meeting to report on progress, a week-long developer meeting to jointly advance the system and a conceptual meeting to set the medium and longer term agenda have been our approach so far. The user meeting is open to everyone interested, but we co-opt developers and users into the two other meetings to make sure, that they remain effective for their tasks. 

The effort documented was funded and supported over the years by too many agencies to name them all, but especially important were ETH Zurich and TU Berlin through the base funding of the current two core groups involved, the Deutsche Forschungsgemeinschaft (DFG), the Schweizer Nationalfonds (SNF), the Swiss Bundesamt für Strassen (ASTRA), Volkswagen AG and the Singaporean National Research Foundation (NRF) through their repeated grants and projects supporting different dissertations through the years. This is gratefully acknowledged by all the researchers involved. 

A special thank goes to the Karen Ettlin, for the very precise and productive copy editing. \ah{we extrapolate here from previous collaborations ;)}. The copy editing is funded by the SNF-grant number ... \ah{this is not yet clear.}

We hope that this book is able to engage the interest of more researchers and engineers to be involved in this joint effort to enable us to provide jointly this framework, which has to be continuously adapted to our policy needs, so that it stays at the forefront of travel behavior modeling.

The editors

Andreas Horni, 	Kai Nagel,		Kay W. Axhausen

\ah{Reihenfolge diskutieren.}

Zürich, January 2015

% ##################################################################################################################
