\chapter*{Editors' Foreword}

\editdone{This text has undergone the professional edit.}

% ##################################################################################################################
Developing complex software for over a decade with a heterogeneous group of engineers and scientists (each with widely different skill levels and expertise across multiple locations around the world) requires dedication and mechanisms unusual for a university environment. 

This book is one of these mechanisms. It allows us, collectively, to take stock and present a coherent, state-of-the-system: for us and anyone interested in this approach. It highlights  basics for the student, who wants to undertake a small first research project as part of his or her degree, provides description of the main functionalities, in detail, for the engineer setting up \gls{matsim} to conduct a policy analysis and, finally, fits the approach into the theoretical background of complex systems in computer science and physics. 

The choice of the e-book format is an advantage, as it allows us to keep the book up-to-date with future chapters, revisions and, if necessary, errata. Equally importantly it allows you, the readers, to select those sections relevant to your needs. 

The book comes at an important time for the system; for most of the first decade, its use was limited to the original developers and users in Berlin and Zürich. It is now much more widely consulted around the world, as we document in the chapter summarizing contributions on \glspl{scenario} so far. 

\Gls{scenario}: this term will occur again and again. In \gls{matsim} context, it is defined as the combination of specific agent populations, their initial \glspl{plan} and \glspl{activitylocation} (home, work, education), the network and facilities where, and on which, they compete in time-space for their slots and modules, \ie behavioral dimensions, which they can adjust during their search for equilibrium. Within these \glspl{scenario}s, the user can experiment and explore with behavioral \gls{utilityfunction} parameters, sampling rate of the population between 1\,\% and 100\,\% and algorithm parameters, \ie share of the sample engaged in \gls{replanning} in any \gls{iteration} and behavioral dimension or exact settings necessary to avoid gridlock due to traffic flow simulation artifacts. The creation of a \gls{scenario} is a substantial effort and the \gls{framework} makes a number of tools available to accelerate it: population synthesizers, network editors, network converters between popular formats and the \gls{matsim} representation (\eg \gls{osm}, \gls{gtfs}), semi-automatic network matching to join information, among others.

A large group of colleagues has been involved and many of them are contributors to this book; this is a list of those involved---both in the past and currently---in Berlin, Singapore and Zürich.  
%
\begin{multicols}{3}
Amit Agarwal \\
Milos Balac  \\
Dr. Michael Balmer\mbox{*} \\
Henrik Becker \\
Joschka Bischoff \\
Patrick Bösch \\
Dr. David Charypar \\
Dr. Nurhan Cetin  \\
Artem Chakirov \\
Dr. Yu Chen \\
Dr. Francesco Ciari \\
Dr. Christoph Dobler \\
Thibaut Dubernet\mbox{*} \\
Dr. Alexander Erath \\
Dr. Matthias Feil \\
Prof. Dr. Gunnar Flötteröd \\
Pieter J. Fourie\mbox{*} \\
Dr. Christian Gloor \\
Dr. Dominik Grether \\
Dr. Jeremy K. Hackney \\
Dr. Johannes Illenberger \\
Prof. Dr. Johan W. Joubert\mbox{*} \\
Ihab Kaddoura \\
Dr. Benjamin Kickhöfer \\
Dr. Gregor Lämmel\mbox{*} \\
Nicolas Lefebvre \\
Dr. Michal Maciejewski \\
Dr. Fabrice Marchal \\
Alejandro Marmolejo \\
Dr. Konrad Meister \\
Dr. Manuel Moyo Oliveros \\
Kirill Müller \\
Dr. Andreas Neumann \\
Dr. Thomas Nicolai \\
Sergio A. Ordóñez Medina \\
Dr. Bryan Raney \\
Dr. Marcel Rieser\mbox{*} \\
Dr. Nadine Rieser-Schüssler \\
Daniel Röder \\
Mohit Shah \\
Lijun Sun \\
Alexander Stahel \\
Prof. Dr. David Strippgen \\
Theresa Thunig \\
Dr. Basil Vitins \\
Michael Van Eggermond \\
Dr. Rashid Waraich\mbox{*} \\
Dominik Ziemke \\
Michael Zilske\mbox{*} \\
\end{multicols}
% 
We hope to acknowledge the contributions of more colleagues from other groups, in future versions of this book and in the software.   

Special thanks to members of the ``committee'' (marked with an \mbox{*}), which makes the final decisions on the \gls{matsim} core, allocation between core and \glspl{contribution} and software engineering of the \gls{framework}.

This committee meeting is a further coordination mechanism; without regular face-to-face meetings, one cannot maintain the necessary trust, communication and understanding. So far, an annual user meeting to report on progress, a week-long developer meeting to jointly advance the system and a conceptual meeting to set the medium and longer-term agenda have worked as our approach. The user meeting is open to everyone interested, but we co-opt developers and users into the two other meetings to ensure they are contributing effectively and efficiently to their tasks. 

A special thanks goes to a number of people, who greatly helped improving this book beyond their own chapters.
Benjamin Kickhöfer's deep knowledge of \gls{matsim}'s mathematical base, particularly its interpretation within the discrete choice framework, made the discussions accompanying the writing of this book very fruitful. 
Thibaut Dubernet's, Marcel Rieser's and Michael Zilske's outstanding expertise on software core development helped us very much and also improved the software structure during the writing of this book.
Marcel Rieser's layout and illustration greatly improved the book's appearance.
Joschka Bischoff's effort to document basic information about every module will greatly help readers make a quick step into respective functionality.
%\gunnar{Ich habe mich aus dem obigen Absatz herausgenommen, hoffe das ist OK.}
%\ah{Muss man das gar nicht mehr erwähnen ;) oder Bescheidenheit? ;)}

The very precise and productive copy editing by Karen Ettlin is gratefully acknowledged. %It is funded by the \gls{snf}-grant number \ah{...} and the \gls{tu} Berlin grant \ah{...}

The reported effort was funded and supported over the years by numerous agencies. Several particularly important sources are: \gls{eth} Zürich and TU Berlin, through the base funding of the two current core groups, the \gls{dfg}, the \gls{snf}, the Swiss \gls{astra}, Volkswagen AG and the \gls{nrf}, through their repeated grants and projects supporting different dissertations over the years. This is gratefully acknowledged by all researchers. 

We hope this book captures the interest of more researchers and engineers and encourages them to get involved in this joint effort. This would enable us to provide this \gls{framework}, which has to be continuously adapted to our policy needs, together and ensure that it stays at the forefront of travel behavior modeling.

The editors

Andreas Horni, 	Kai Nagel,	Kay W. Axhausen

%\ah{Reihenfolge diskutieren.}

Zürich, October~2015

% ##################################################################################################################
% \ah{We merge somehow, ``Foreword'', ``Preface'' and ``Acknowledgements'', but it is much nicer like that, than split}
%\url{http://www.thebookdesigner.com/2009/09/parts-of-a-book/}:
%``Foreword: Usually a short piece written by someone other than the author, the Foreword may provide a context for the main work. Remember that the Foreword is always signed, usually with the author’s name, place and date.
%
%Preface: Written by the author, the Preface often tells how the book came into being, and is often signed with the name, place and date, although this is not always the case.''

