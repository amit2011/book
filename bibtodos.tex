%%\ahtodo{\kai{Andreas, viele der Referenzen sind zu kurz.  Ich vermute, dass sich viele von denen auf das URL-Feld verlassen, aber das ist ausgeblendet.  Ich mache das in der Tat immer so, dass ich das URL-Feld ausblende und es lieber z.B. in note=\{...\} als \url{...} eingebe, weil ich dann mehr Kontrolle habe, die ``tiefen'' Links zu vermeiden (die ich ja auch anderswo nicht mag).  Wenn es gar nicht anders geht, dann könnten wir die URL-Felder doch akzeptieren, und ich überlege mir für vsp.bib ein Skript, um sie loszuwerden.  Schaust Du bitte mal?  Danke.}
%%\\
%%\kai{Ok, weiteres Nachschauen hat folgendes ergeben: (1) ist wohl darauf zurückzuführen, dass der template\_ivt-eng bibtex style die notes nicht berücksichtigt.  Bei unseren Einträgen funktioniert das nicht ... (2) template\_ivt-eng - with - notes kann ich wg. der blanks nicht verwenden. (3) Nach rename der Datei funktioniert es immer noch nicht, irgendeine infinite loop.  (4) das dbj file ist nicht dabei, ich kann es nicht selber reparieren. (5) ok, habe jetzt erstmal vsp.bbl ersatzweise genommen.  Jetzt gibt es die notes, aber dafür URLs nicht mehr (die bei IVT Einträgen wichtig sind). (6) daher jetzt vsp.bst neu erzeugt mit URLs.}
%%\\
%%\kai{Habe jetzt doch template ivt eng with notes ans Laufen gebracht.  Dennoch waere das zugehoerige dbj-File hilfreich.}
%%}

\ahtodo{BE14 capitalize swedish}


