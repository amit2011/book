% ##############################################################################################################################
\chapter*{Symbols and Typographic Conventions}
\label{ch:conventionsSymbols}
% ##############################################################################################################################

\ah{Sollte das in (verschiedene Glossaries rein?}

\section*{Symbols}
\label{sec:symbols}

% ============================================================================================
\subsection*{Variables}
\ah{Bez�glich Gross- und Kleinschreibung habe ich jetzt grad ein bisschen ein Durcheinader mit Statistik, Algebra und Analysis. U wie Utility ist bei uns gross. t ist aber klein. Meist wird U aber als Funktion angegeben. Brauchen wir hier Konsistenz bis zum Kopfschmerz oder lassen wir die gewachsenen Bezeichnungen so?}

\kai{Ich habe kein Problem damit, dass Funktionen, z.B.\ $U(...)$, auch gro� geschrieben werden k�nnen.  Ich habe auch kein Problem damit, dass hier nicht jedes Mal die vollst�ndige Argumentliste angegeben wird, oder es sogar gar nicht als Funktion notiert wird (also einfach $U = ...$).}


\begin{tabular}{l l}
	$c$ & monetary costs \\
  $d$ & distance \\
  $t$ & time \\
  $U$ & utility variable ($V + \varepsilon$) \\
  $V$ & systematic component of utility variable \\
  $S$ & score ($=$ the un-interpreted \acrshort{matsim} value)\\
  $\beta$ & utility function coefficient \\
  $\hat{\beta}$ & estimated utility function coefficient \\
  $\varepsilon$ & random component of utility variable \kai{note that this is {\tt vareps} and not {\tt epsilon}}\\
  $\phi$ & replanning share \\
  $\sigma$ & scale parameter of the multinomial logit model \gunnar{$\mu$}\kai{agree with gunnar.  $\sigma$ is something that grows with more variation.} \\
\kai{???} & \kai{need some index to denote iterations.  Typically, this is $n$, but this may now be (reasonably) reserved for agents} \\

  
\end{tabular}

% ============================================================================================
\subsection*{Indices and Subscripts}
\begin{tabular}{l l}
  $i$ & index of plan activities\\
  $j$ & index of agents \gunnar{index agents by $n$ (like decision makers in discrete choice)}\kai{agree with Gunnar}\\
  $k$ & index of plans \gunnar{index plans by $i$ (like alternatives in discrete choice)}\kai{agree with Gunnar} \\
  $l$ & index of locations/facilities/alternatives\kai{suggestion to use {\tt ell} ($\ell$) instead of {\tt l} ($l$)} \\
\end{tabular}

\ah{noch abstimmen mit 
- F�r Verhaltensmodelle: Ben-Akiva and Lerman (1985). \\
- F�r Netzwerkmodelle: Daganzo and followers. \\
- F�r assignment eventuell Nagel und Fl�tter�d (2012). \\
(G. Fl�tter�d)
}

% ##############################################################################################################################
\section*{Typographic Conventions}
\label{sec:typogr-conv}

blabla

% ##############################################################################################################################

% Local Variables:
% mode: latex
% mode: reftex
% mode: visual-line
% TeX-master: "../main"
% comment-padding: 1
% fill-column: 9999
% End: 

