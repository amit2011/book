% ##############################################################################################################################
\chapter*{Symbols and Typographic Conventions}
\label{ch:conventionsSymbols}
% ##############################################################################################################################
\section*{Symbols}
\label{sec:symbols}

% ============================================================================================
\subsection*{Variables}
\begin{tabular}{l l}
	$c$ & monetary costs \\
  $d$ & distance \\
  $t$ & time \\
  $U$ & utility variable ($V + \varepsilon$) \\
  $V$ & systematic component of utility variable \\
  $S$ & score ($=$ the un-interpreted \acrshort{matsim} value)\\
  $\beta$ & utility function coefficient \\
  $\hat{\beta}$ & estimated utility function coefficient \\
  $\varepsilon$ & random component of utility variable \\ %\kai{note that this is {\tt vareps} and not {\tt epsilon}}
  $\varphi$ & replanning share \\
  $\mu$ & scale parameter of the multinomial logit model \\  
\end{tabular}

% ============================================================================================
\subsection*{Indices and Subscripts}
\begin{tabular}{l l}
  $i$ & index of plans \\
  $k$ & index of iterations \\
  $n$ & index of agents \\
  $q$ & index of plan activities\\
  $\ell$ & index of activity locations/facilities \\
\end{tabular}

% ##############################################################################################################################
%\begin{tabular}{l l}
%	$c$ & monetary costs \\
%  $d$ & distance \\
%  $t$ & time \\
%  $U$ & utility variable ($V + \varepsilon$) \\
%  $V$ & systematic component of utility variable \\
%  $S$ & score ($=$ the un-interpreted \acrshort{matsim} value)\\
%  $\beta$ & utility function coefficient \\
%  $\hat{\beta}$ & estimated utility function coefficient \\
%  $\varepsilon$ & random component of utility variable \kai{note that this is {\tt vareps} and not {\tt epsilon}}\\
%  $\phi$ & replanning share \\
%  $\sigma$ & scale parameter of the multinomial logit model \gunnar{$\mu$}\kai{agree with gunnar.  $\sigma$ is something that grows with more variation.} \\
%\kai{???} & \kai{need some index to denote iterations.  Typically, this is $n$, but this may now be (reasonably) reserved for agents} \\
%
%  
%\end{tabular}
%
%% ============================================================================================
%\subsection*{Indices and Subscripts}
%\begin{tabular}{l l}
%  $i$ & index of plan activities\\
%  $j$ & index of agents \gunnar{index agents by $n$ (like decision makers in discrete choice)}\kai{agree with Gunnar}\\
%  $k$ & index of plans \gunnar{index plans by $i$ (like alternatives in discrete choice)}\kai{agree with Gunnar} \\
%  $l$ & index of locations/facilities/alternatives\kai{suggestion to use {\tt ell} ($\ell$) instead of {\tt l} ($l$)} \\
%\end{tabular}
%
%\ah{noch abstimmen mit 
%- Für Verhaltensmodelle: Ben-Akiva and Lerman (1985). \\
%- Für Netzwerkmodelle: Daganzo and followers. \\
%- Für assignment eventuell Nagel und Flötteröd (2012). \\
%(G. Flötteröd)
%}

% ##############################################################################################################################
\section*{Typographic Conventions}
\label{sec:typogr-conv}
The \lstinline|listing-format| is used for text that you typically see when you run \gls{matsim}, \ie program snippets, commands, on-screen computer output, input and output file names and content, and configurations to be specified in the \gls{configfile}. Larger snippets are shown as complete listings.
\begin{lstlisting}
... main( ... ) {
    // construct the config object:
    Config config = ConfigUtils.xxx(...) ;
    config.xxx().setYyy(...) ;
    ...
}
\end{lstlisting}

Important passages are \emph{emphasized}. 

%<Carets> denote keys that have to be pressed or buttons that have to be pressed.

Vertical bar $\lvert$ is a separator for mutually exclusive items. For example: ``\lstinline$KeepLastSelected | BestScore | SelectExpBeta$''
%``\lstinline$KeepLastSelected | BestScore | SelectExpBeta | ChangeExpBeta | SelectRandom | SelectPathSizeLogit$''

Math mode, \eg $x=42$ is used for mathematical terms.

\glsresetall
Acronyms are given with the abbreviation and the description following in parenthesis (\eg 
\gls{matsim}) on first occurrence, later only the abbreviation (\eg \gls{matsim}) is given.

% ##############################################################################################################################

% Local Variables:
% mode: latex
% mode: reftex
% mode: visual-line
% TeX-master: "../main"
% comment-padding: 1
% fill-column: 9999
% End: 

