%##############################################################################################################################

%When to gls{term}?:
%
%There seem not to be general rules of thumb for how often and when to gls() a term.
%I would suggest, that we do NOT gls() every occurrence of a term. We will have dozens of pages referenced in the glossary which is kind of useless.
%Rather I would suggest, that we gls() the first occurrence in every book part 
%and additionally we add some more distinct gls(), where the term is actually the \emph{subject} of a statement and not just an auxiliary word to explain something else.
%I know with some terms, such as MATSim, this is difficult, but as a rough guideline it might be productive. 

% -------------
% When to use "newacronym" and "newglossaryentry"? 
% When used (referenced) for the first time both entries and accronyms are replaced by the short description in the text (not the one in the []-brackets). 
% For all further referencing only the name is shown.
% Accronyms, additionally show the abbreviation in brackets for the first time.

%##############################################################################################################################

\newglossaryentry{aggregate}{
    name=Aggregate,
    description={...},
		text={aggregate},
		plural={aggregates},
}

\newglossaryentry{activity}{
    name=Activity,
    description={Modern transport planning assumes that "\emph{travel demand is derived from activity demand}" \citep[][]{Jones_HensherStopher_1979, Bowman_TEC_2009_1, Bowman_TEC_2009_2, BhatKoppelman_HallRW_2003, EttemaTimmermans_1997, BowmanBenAkiva_ABTFCEngelke_1996, BowmanBenAkiva_TransResA_2001}. People travel because they want to perform a certain activity, and thus activities have become the central element of modeling},
		text={activity},
		plural={activities},
%		descriptionplural={...},
}

\newglossaryentry{activitybased}{
    name=Activity-based,
    description={...},
		text={activity-based}
}

\newglossaryentry{activitylocation}{
    name=Activity location,
    description={People perform their activities at activity locations, which can be as small as one single building or large zones. In MATSim activity locations are often further specified by using the \gls{facility} object, which besides others define open times},
		text={activity location},
		plural={activity locations},
}

\newglossaryentry{agent}{
    name=Agent,
    description={According to \citet[][p.21]{Wooldridge_2009} an agent is "\emph{is a computer system that is situated in some environment, and that is capable of autonomous action in this environment in order to meet its delegated objectives.}"},
		text={agent},
		plural={agents},
}

\newglossaryentry{algorithm}{
    name=Algorithm,
    description={...},
		text={algorithm},
		plural={algorithms},
}

\newglossaryentry{contribution}{
    name=Contribution,
    description={A MATSim contribution (short "contrib") is an \gls{extension} that made it into an official MATSim release},
		text={contribution},
		plural={contributions},
}

\newglossaryentry{configfile}{
    name=Configuration file,
    description={often just referred to as \lstinline|config file| or as \lstinline|config.xml| ...},
		text={config file},
		plural={config files},
}

\newglossaryentry{core}{
    name=Core,
    description={...},
		text={core}
}

\newglossaryentry{cpp}{
    name=C++,
    description={Cplusplus is an object-oriented programming language with full control of memory management},
		text={C++}
}

\newglossaryentry{csharp}{
	name=CSharp,
	description={...},
	text={C\#}
}

\newglossaryentry{disaggregate}{
    name=Disaggregate,
    description={...},
		text={disaggregate},
		plural={disaggregates},
}

\newglossaryentry{eclipse}{
    name=Eclipse,
    description={The standard integrated development environment (IDE) used by the MATSim developers},
		text={Eclipse}
}

\newglossaryentry{equilibrium}{
    name=Equilibrium,
    description={...},
		text={equilibrium},
		plural={equilibria},
}

\newglossaryentry{event}{
    name=Event,
    description={...},
		text={event},
		plural={events},
}

\newglossaryentry{extension}{
    name=Extension,
    description={MATSim extensions are written by \gls{api} users. An extension is code extending MATSim base functionality and hooking to MATSim via the extension points provided in the \gls{api}. If extensions are included in a MATSim release then they are called MATSim contributions},
		text={extension},
		plural={extensions},
}

\newglossaryentry{facility}{
    name=Facility,
    description={Facilities are optional elements in MATSim to further specify an \gls{activitylocation}},
		text={facility},
		plural={facilities},
}

\newglossaryentry{framework}{
    name=Framework,
    description={A framework provides generic functionality and application specific software. It is selectively changed by user code. MATSim is currently a framework},
		text={framework},
		plural={framework},
}

\newglossaryentry{iteration}{
    name=Iteration,
    description={Numerical equilibrium search methods, such as MATSim are iterative. A \gls{matsimrun} is thus composed of a configurable number of iterations},
		text={iteration},
		plural={iterations},
}

\newglossaryentry{java}{
    name=Java,
    description={Java is a modern object-oriented programming language and is cross-platform capable due to its bytecode run in any Java virtual machine},
%		text={Java programming language}
% I find this ``too heavyweight''.  Need to review grammar!!!
		text={Java}
}

\newglossaryentry{javadoc}{
    name=Javadoc,
    description={Javadoc is source code documentation that is compiled from javadoc annotations in the dource files},
		text={javadoc}
}

\newglossaryentry{jenkins}{
    name=Jenkins,
    description={A software tool for continous integartion},
		text={Jenkins}
}

\newglossaryentry{leg}{
    name=Leg,
    description={A leg is part of a \gls{trip} performed with a specific mode. In transport planning this is often called a \gls{stage}},
		text={leg},
		plural={legs},
}

\newglossaryentry{library}{
    name=Library,
    description={A library is a set of routines providing services to independent programs. Usually it is not executable},
		text={library},
		plural={libraries},
}

\newglossaryentry{link}{
    name=Link,
    description={A link is part of a MATSim network representing the streets},
		text={link},
		plural={links},
}

\newglossaryentry{mac}{
		name=Mac,
    description={...},
		text={mac}
}

\newglossaryentry{matsimrun}{
    name=MATSim run,
    description={A MATSim run is composed of a configurable number of \glspl{iteration} and ends with an equilibrium solution of transport supply and demand},
		text={MATSim run}	
}

\newglossaryentry{maven}{
    name=Maven,
    description={A build automation tool tailored to \gls{java}},
		text={Apache Maven}
}

\newglossaryentry{microcensus}{
    name=Microcensus,
    description={Swiss microcensus...},
		text={microcensus},
		plural={microcensi},
}

\newglossaryentry{microsimulation}{
    name=Microsimulation,
    description={...},
		text={microsimulation},
		plural={microsimulations},
}

\newglossaryentry{model}{
    name=Model,
    description={A model reduces a \gls{system} to the aspects relevant for understanding or solving a specific problem},
		text={model}	
}

\newglossaryentry{module}{
    name=Module,
    description={According to the Merriam-Webster (\url{http://www.merriam-webster.com}), a module is "one of a set of parts that can be connected or combined to build or complete something" or more specifically "a part of a computer or computer program that does a particular job". That is, "module" is not a very specific term, and in consequence modules exist in MATSim at many levels},
		text={module},
		plural={modules},
}

\newglossaryentry{multimodal}{
    name=Multi-Modal,
    description={...},
		text={multi-modal}	
}

\newglossaryentry{node}{
    name=Node,
    description={A node is part of a MATSim network representing intersections. Note that intersections are not modeled explicitly in MATSim, i.e.,\,cars do not interact at intersections},
		text={node},
		plural={nodes},
}

\newglossaryentry{objectivefunction}{
    name=Objective function,
    description={...},
		text={objective function}
}

\newglossaryentry{parameter}{
    name=Parameter,
    description={Program parameter vs. statistical parameters ...},
		text={parameter},
		plural={parameters},
}

\newglossaryentry{plan}{
    name=Plan,
    description={An agent's plan contains his or her day schedule and after run completion an associated \gls{score}},
		text={plan},
		plural={plans},
}

\newglossaryentry{replanning}{
    name=Replanning,
    description={MATSim replanning denotes the stage where agents (randomly) modify their plans},
		text={replanning}	
}

\newglossaryentry{scenario}{
    name=Scenario,
    description={A scenario...},
		text={scenario}
}

\newglossaryentry{score}{
    name=Score,
    description={After execution in the infrastructure the agents' day plans are evaluated through an individual \gls{objectivefunction}, the MATSim scoring function},
		text={score}
}

\newglossaryentry{scoring}{
    name=Scoring,
    description={see score},
		text={score}
}

\newglossaryentry{senozon}{
    name=Senozon AG,
    description={A spin-off company founded by two core programers of MATSim},
		text={Senozon AG}
}

\newglossaryentry{simulation}{
    name=Simulation,
    description={A simulation evaluates a \gls{model} ...},
		text={simulation},
		plural={simulations},
}

\newglossaryentry{sourceforge}{
    name=SourceForge,
    description={A public source code repository, see \url{http://sourceforge.net/}},
		text={SourceForge}
}

\newglossaryentry{study}{
    name=Study,
    description={...},
		text={study}
}

\newglossaryentry{stage}{
    name=Stage,
    description={A leg is part of a trip performed with a single mode. In MATSim called \gls{leg}},
		text={stage}
}

\newglossaryentry{system}{
    name=System,
    description={...},
		text={system}
}

\newglossaryentry{teleport}{
    name=Teleport,
    description={...},
		text={teleport}
}

\newglossaryentry{teleportation}{
    name=Teleportation,
    description={see teleport},
		text={teleportation},
		plural={teleportations},
}

\newglossaryentry{toolkit}{
    name=Toolkit,
    description={A toolkit is a collection of tools},
		text={toolkit}
}

\newglossaryentry{trafficassignment}{
    name=Traffic assignment,
    description={...},
		text={traffic assignment}
}

\newglossaryentry{trafficflowmodel}{
    name=Traffic flow model,
    description={...},
		text={trafficflowmodel}
}

\newglossaryentry{transit}{
    name=Transit,
    description={...},
		text={transit}
}

\newglossaryentry{trip}{
    name=Trip,
    description={A trip connects two \glspl{activity} and is composed of multiple \glspl{leg}},
		text={trip},
		plural={trips},
}

\newglossaryentry{unix}{
		name=Unix,
    description={...},
		text={unix}
}

\newglossaryentry{utility}{
    name=Utility,
    description={...},
		text={utility}
}

\newglossaryentry{utilityfunction}{
    name=Utility function,
    description={...},
		text={utilityfunction},
		plural={utility functions},
}

\newglossaryentry{windows}{
		name=Windows,
    description={...},
		text={Windows}
}
% ========================================================================

%\newglossaryentry{transit}{
    %name=Transit,
    %description={...},
		%text={transit}
%}

% ##############################################################################################################################
