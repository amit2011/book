%##############################################################################################################################

%When to gls{term}?:
%
%There seem not to be general rules of thumb for how often and when to gls() a term.
%I would suggest, that we do NOT gls() every occurrence of a term. We will have dozens of pages referenced in the glossary which is kind of useless.
%Rather I would suggest, that we gls() the first occurrence in every book part 
%and additionally we add some more distinct gls(), where the term is actually the \emph{subject} of a statement and not just an auxiliary word to explain something else.
%I know with some terms, such as MATSim, this is difficult, but as a rough guideline it might be productive. 

% -------------
% When to use "newacronym" and "newglossaryentry"? 
% When used (referenced) for the first time both entries and accronyms are replaced by the short description in the text (not the one in the []-brackets). 
% For all further referencing only the name is shown.
% Accronyms, additionally show the abbreviation in brackets for the first time.

%##############################################################################################################################

\newglossaryentry{aggregate}{
    name=Aggregate,
    description={A combination of objects forming a total},
		text={aggregate},
		plural={aggregates},
}

\newglossaryentry{activity}{
    name=Activity,
    description={The central element of modern activity-based modeling (see below)},
		text={activity},
		plural={activities}
}

\newglossaryentry{activitybased}{
    name=Activity-based,
    description={Modern transport planning assumes that ``\emph{travel demand is derived from activity demand}'' \citep[][]{Jones_HensherStopher_1979, Bowman_TEC_2009_1, Bowman_TEC_2009_2, BhatKoppelman_HallRW_2003, EttemaTimmermans_1997, BowmanBenAkiva_ABTFCEngelke_1996, BowmanBenAkiva_TransResA_2001}. People travel because they want to perform a certain activity, which is best captured by activity-based models with activities being the central element of modeling},
		text={activity-based}
}

\newglossaryentry{activitylocation}{
    name=Activity location,
    description={People perform their activities at activity locations, which can be as small as one single building or large zones. In MATSim activity locations are often further specified by using the \gls{facility} object, which besides others define open times},
		text={activity location},
		plural={activity locations},
}

\newglossaryentry{agent}{
    name=Agent,
    description={According to \citet[][p.21]{Wooldridge_2009} an agent is ``\emph{is a computer system that is situated in some environment, and that is capable of autonomous action in this environment in order to meet its delegated objectives.}''},
		text={agent},
		plural={agents},
}

%\newglossaryentry{ai}{
    %name=Artificial Intelligence,
    %description={...},
		%text={Artificial Intelligence},
		%plural={...},
%}
%seems that we do not use this often enough to defined the acronym in chapter 3 to not re-use it before chapter 20. kai, jan'15

\newglossaryentry{algorithm}{
    name=Algorithm,
    description={A set of operations to solve a specific problem},
		text={algorithm},
		plural={algorithms},
}

\newglossaryentry{arcgis}{
    name=ArcGIS,
    description={ESRI's geographic information system},
		text={ArchGIS},
}

\newglossaryentry{contribution}{
    name=Contribution,
    description={An \gls{extension} contributed by \gls{matsim} community and hosted in \url{http://www.matsim.org/docs/extensions}. It residues in the \lstinline|org.matsim.contrib|},
		text={contribution},
		plural={contributions},
}

\newglossaryentry{configfile}{
    name=Configuration file,
    description={The main configuration screw for MATSim often just referred to as \lstinline|config file| or as \lstinline|config.xml| ...},
		text={config file},
		plural={config files},
}

\newglossaryentry{config}{
    name=Configuration object,
    description={The object in the \gls{matsim} code that contains the configuration options.  It can be modified by the \gls{configfile}, but also by other mains, in particular by scripts-in-\gls{java}},
    text={config},
    plural={configs},
}


\newglossaryentry{core}{
    name=Core,
    description={The central parts of \gls{matsim} main distribution, residing in packages \item \lstinline{org.matsim.api.*}, \lstinline{org.matsim.core.*}, and \lstinline{tutorial.*}},
		text={core}
}

\newglossaryentry{cpp}{
    name=C++,
    description={An object-oriented programming language with full control of memory management},
		text={C++}
}

\newglossaryentry{csharp}{
	name=C\#,
	description={An object-oriented .NET programming language},
	text={C\#}
}

\newglossaryentry{disaggregate}{
    name=Disaggregate,
    description={In transport modeling: using the individual as the basic modeling unit (see e.g., \citet[][]{BenAkiva_TRR_1974, DomencichMcFadden_1975, BenAkivaLerman_1985}, \citet[][p.20]{OrtuzarWillumsen_2001})},
		text={disaggregate},
		plural={disaggregates},
}

\newglossaryentry{eclipse}{
    name=Eclipse,
    description={The standard integrated development environment (IDE) used by the MATSim developers},
		text={Eclipse}
}

\newglossaryentry{equilibrium}{
    name=Equilibrium,
    description={A system state where are competing forces are balanced},
		text={equilibrium},
		plural={equilibria},
}

\newglossaryentry{event}{
    name=Event,
    description={Small pieces of information reported by the mobsim and describing the action of a simulation object at a specific time},
		text={event},
		plural={events},
}

\newglossaryentry{extension}{
    name=Extension,
    description={Core MATSim only uses \gls{configfile}, population, and network corresponding to the book's part~I MATSim. An extension is any code that extends this core MATSim corresponding to this book's part~II. They hook to MATSim via the extension points described in Chapter~\ref{ch:extensionpoints}},
		text={extension},
		plural={extensions},
}

\newglossaryentry{facility}{
    name=Facility,
    description={An optional element in MATSim to further specify an \gls{activitylocation}},
		text={facility},
		plural={facilities},
}

\newglossaryentry{framework}{
    name=Framework,
    description={A software concept, providing generic functionality and application specific software. It is selectively changed by user code. MATSim is currently a framework},
		text={framework},
		plural={framework},
}

\newglossaryentry{geocoding}{
    name=Geocoding,
    description={Adding geographic coordinates to locations identified by addresses},
		text={geocoding}
}

\newglossaryentry{git}{
    name=Git,
    description={A free and open source distributed version control system},
        text={Git},
}
% Git, not GIT. kai, aug'15


\newglossaryentry{github}{
    name=GitHub,
    description={A web-based Git repository hosting service, see \url{https://github.com/}},
		text={GitHub}
}

\newglossaryentry{googleearth}{
name=Google Earth,
description={Google's virtual globe},
text={Google Earth},
}

\newglossaryentry{id}{
    name=Identifier,
    description={A name that labels an object in a unique way},
		text={ID},
		plural={IDs},
}

\newglossaryentry{iteration}{
    name=Iteration,
    description={Numerical equilibrium search methods, such as MATSim are iterative. A \gls{matsimrun} is thus composed of a configurable number of iterations},
		text={iteration},
		plural={iterations},
}

\newglossaryentry{java}{
    name=Java,
    description={A modern object-oriented, cross-platform programming language run in virtual machines},
%		text={Java programming language}
% I find this ``too heavyweight''.  Need to review grammar!!!
		text={Java}
}

\newglossaryentry{javadoc}{
    name=Javadoc,
    description={Source code documentation that is compiled from javadoc annotations in the source files},
		text={javadoc}
}

\newglossaryentry{jenkins}{
    name=Jenkins,
    description={A software tool for continuous integration},
		text={Jenkins}
}

\newglossaryentry{largescale}{
    name=Large-scale,
    description={Denoting large extended simulation scenarios, often modeling complete cities or even countries},
		text={large-scale}
}

\newglossaryentry{leg}{
    name=Leg,
    description={A plan element, being part of a \gls{trip} performed with a specific mode. In transport planning this is often called a \gls{stage}},
		text={leg},
		plural={legs},
}

\newglossaryentry{library}{
    name=Library,
    description={A set of routines providing services to independent programs. Usually it is not executable},
		text={library},
		plural={libraries},
}

\newglossaryentry{link}{
    name=Link,
    description={A network component representing the streets},
		text={link},
		plural={links},
}

\newglossaryentry{linux}{
    name=Linux,
    description={A unix-like operating system released by Linus Torvalds by end of 1991},
		text={Linux},
}

\newglossaryentry{logsum}{
    name=Logsum,
    description={The \gls{emu} for a user that has several options. Computed as the logarithm of the sum of exponential functions},
		text={logsum},
}

\newglossaryentry{mac}{
		name=Mac OS,
    description={The operating system by Apple Inc. developed for their Macintosh computer systems},
		text={Mac}
}

\newglossaryentry{matsimmain}{
    name=MATSim Main Distribution,
    description={The ``matsim'' part of the \gls{svn} repository maintained by the \gls{matsim} team},
		text={MATSim main distribution}
}

\newglossaryentry{matsimrun}{
    name=MATSim run,
    description={A set of a configurable number of \glspl{iteration} and ending with an equilibrium solution of transport supply and demand},
		text={MATSim run}	
}

\newglossaryentry{maven}{
    name=Maven,
    description={A build automation tool tailored to \gls{java}},
		text={Apache Maven}
}

\newglossaryentry{microcensus}{
    name=Microcensus,
    description={A cross-sectional survey containing 30\,000 person day diaries, representing a central data source for Swiss transport modeling},
		text={microcensus},
		plural={microcensi},
}

\newglossaryentry{microsimulation}{
    name=Microsimulation,
    description={The modeling of the temporal development of a real-world system or process by explicitly considering the interactions of micro units such as individuals or vehicles. For concise definitions and further information see \eg \citet[][Section 2]{Miller_ABTFC_1996} or \citet[][p.3]{Banks_2001}, \citet[][]{Bossel_2004} or \citet[][]{Orcutt_RESQ_1957}, who is often referred to as the inventor of \gls{microsimulation}. \\
According to this definition, only the program components that model transition processes are strictly speaking \gls{microsimulation}. In \gls{matsim} for example, only the \gls{mobsim} is actually a simulation. However, the delineation is difficult and, thus, in the transport planning community, the term \emph{microsimulation} is ambiguously used. Sometimes it actually denotes only the simulation of traveling persons and vehicles in the assignment step---as a replacement of volume-delay functions in aggregate models \citep[see e.g.,][p.508]{NagelBarrett1997feedback}. More often, it additionally includes the preceding choice processes \citep[][]{Kitamura_TMIP_1996, LiuEtAl_TransResA_2006}. In this book we use \gls{microsimulation} in this sense, which includes the activity-based demand modeling part and the dynamic traffic assignment (for a detailed discussion of combination of these two parts see \citet[][p.10ff]{Balmer_PhDThesis_2007})
		},
		text={microsimulation},
		plural={microsimulations},
}

\newglossaryentry{model}{
    name=Model,
    description={A universal concept reducing a real system to the aspects relevant for understanding or solving a specific problem},
		text={model}	
}

\newglossaryentry{module}{
    name=Module,
    description={According to the Merriam-Webster (\url{http://www.merriam-webster.com}), a module is ``one of a set of parts that can be connected or combined to build or complete something'' or more specifically ``a part of a computer or computer program that does a particular job''. That is, ``module'' is not a very specific term, and in consequence modules exist in MATSim at many levels},
		text={module},
		plural={modules},
}

\newglossaryentry{multimodal}{
    name=Multimodal,
    description={Combining different means of transport},
		text={multimodal}	
}

\newglossaryentry{navteq}{
    name=NAVTEQ,
    description={A geographical information system data provider, in particular for navigation maps},
		text={NAVTEQ}	
}

\newglossaryentry{node}{
    name=Node,
    description={An element of a MATSim network representing intersections. Note that intersections are not modeled explicitly in MATSim, \ie cars do not interact at intersections},
		text={node},
		plural={nodes},
}

\newglossaryentry{objectivefunction}{
    name=Objective function,
    description={A central element in optimization problems, amongst others. An objective function, sometimes also called loss or cost function, is mapping of candidate solutions onto a real number},
		text={objective function}
}

\newglossaryentry{omnitrans}{
    name=OmniTRANS,
    description={A transport Modeling Software Platform},
		text={OmniTRANS}
}

\newglossaryentry{osmosis}{
        name=Osmosis,
        description={Command line \gls{java} application for processing \gls{osm} data.  See \url{http://wiki.openstreetmap.org/wiki/Osmosis}},
        text={Osmosis},
}

\newglossaryentry{parameter}{
    name=Parameter,
    description={Program parameter vs. statistical parameters ...},
		text={parameter},
		plural={parameters},
}

\newglossaryentry{pathsizelogitmodel}{
    name=Path Size Logit Model,
    description={An extension of the \gls{mnl} model to correct overestimation of overlapping paths in route choice problems},
		text={path size logit model},
}

\newglossaryentry{plan}{
    name=Plan,
    description={The agent's day schedule and after run completion an associated \gls{score}},
		text={plan},
		plural={plans},
}

\newglossaryentry{qsim}{
    name=QSim,
    description={The standard MATSim \gls{mobsim}},
		text={QSim}	
}

\newglossaryentry{replanning}{
    name=Replanning,
    description={The stage when agents (randomly) modify their plans},
		text={replanning}	
}

\newglossaryentry{scenario}{
    name=Scenario,
    description={In MATSim context, a scenario is defined as the combination of specific agent populations, their initial plans and activity locations (home, work, education), the network and facilities where, and on which, they compete in time-space for their slots and modules, \ie behavioral dimensions, which they can adjust during their search for equilibrium.},
		text={scenario}
}

\newglossaryentry{score}{
    name=Score,
    description={After execution in the infrastructure the agents' day plans are evaluated through an individual \gls{objectivefunction}, the MATSim scoring function},
		text={score}
}

\newglossaryentry{scoring}{
    name=Scoring,
    description={see score},
		text={score}
}

\newglossaryentry{senozon}{
    name=Senozon AG,
    description={A spin-off company founded by two core developers of MATSim},
		text={Senozon AG}
}

\newglossaryentry{simulation}{
    name=Simulation,
    description={Evaluating a \gls{model} capturing the temporal development of a real-world system or process},
		text={simulation},
		plural={simulations},
}

\newglossaryentry{sourceforge}{
    name=SourceForge,
    description={A public source code repository, see \url{http://sourceforge.net/}},
		text={SourceForge}
}

\newglossaryentry{study}{
    name=Study,
    description={The basic organizational unit of research in empirical science. Comparable to the experiment in the natural sciences},
		text={study}
}

\newglossaryentry{stage}{
    name=Stage,
    description={A leg is part of a trip performed with a single mode. In MATSim called \gls{leg}},
		text={stage}
}

\newglossaryentry{sustaincity}{
    name=SustainCity,
    description={A project addressing the modeling and computational issues of integrating modern mobility simulations with the latest \gls{microsimulation} land use models, see \url{http:\\www.sustaincity.org}},
		text={SustainCity}
}

\newglossaryentry{teleport}{
    name=Teleport,
    description={Moving vehicles form origin to destination according to a predefined speed without considering interactions in the network},
		text={teleport}
}

\newglossaryentry{teleportation}{
    name=Teleportation,
    description={see teleport},
		text={teleportation},
		plural={teleportations},
}

\newglossaryentry{teleported}{
    name=Teleported,
    description={see teleport},
		text={teleported},
		plural={teleported},
}

\newglossaryentry{toolkit}{
    name=Toolkit,
    description={A collection of tools},
		text={toolkit}
}

\newglossaryentry{trafficassignment}{
    name=Traffic assignment,
    description={Traditionally, this is the last step of the four-step model calculating how trips distribute over the different routes between \glspl{od}-pairs. Dynamic traffic assignment additionally considers respective departure time choices},
		text={traffic assignment}
}

\newglossaryentry{trafficflowmodel}{
    name=Traffic flow model,
    description={...},
		text={trafficflowmodel}
}

\newglossaryentry{transit}{
    name=Transit,
    description={Public transport. In German speaking countries (``Transit Verkehr'') ambiguously used for traffic going through cities or countries neither having their origin or destination there},
		text={transit}
}

\newglossaryentry{trip}{
    name=Trip,
    description={The connection between two \glspl{activity} composed of multiple \glspl{leg}},
		text={trip},
		plural={trips},
}

\newglossaryentry{unix}{
		name=Unix,
    description={An operating system developed in 1970s at the Bell Labs research center},
		text={unix}
}

\newglossaryentry{utility}{
    name=Utility,
    description={A central concept in economics representing satisfaction through goods consumption. The MATSim score can be interpreted in utility units},
		text={utility}
}

\newglossaryentry{utilityfunction}{
    name=Utility function,
    description={A MATSim agent's \gls{objectivefunction}. See also \gls{utility}},
		text={utility function},
		plural={utility functions},
}

\newglossaryentry{via}{
    name=Via,
    description={The \gls{senozon} visualizer},
		text={Via}
}

\newglossaryentry{windows}{
		name=Windows,
    description={An operating system developed by Microsoft an first released in 1985},
		text={Microsoft Windows}
}
% ========================================================================

%\newglossaryentry{transit}{
    %name=Transit,
    %description={...},
		%text={transit}
%}

% ##############################################################################################################################
